\chapter{}

\section*{\S5. Artinian modules}

\begin{definition}
	Let $R$ be a ring. A module $N$ is \textbf{artinian} if every decreasing sequence 
	$N_1\supseteq N_2\supseteq\cdots$ of submodules of $N$ stabilizes, that is
	there exists $n\in\Z_{>0}$ such that 
	$N_n=N_{n+k}$ for all $k\in\Z_{>0}$.
\end{definition}

Let $X$ be a set and $\mathcal{S}$ be a set of subsets of $X$. 
We say that $A\in\mathcal{S}$ is a \textbf{minimal element} of $\mathcal{S}$
if there is no $Y\in\mathcal{S}$ such that $Y\subsetneq A$. 

\begin{proposition}
\label{pro:artinian_minimal}
	A module $N$ is artinian if and only if every non-empty subset of submodules of $N$ 
	contains a minimal element. 
\end{proposition}

\begin{proof}
	Assume that $N$ is artinian. Let $\mathcal{S}$ be the non-empty set of submodules of $N$. 
	Suppose that $\mathcal{S}$ has no minimal element and let $N_1\in\mathcal{S}$. 
	Since $N_1$ is not minimal, there exists 
	$N_2\in\mathcal{S}$ such that $N_1\supsetneq N_2$. Now assume the 
	submodules 
	\[
	N_1\supsetneq N_2\supseteq\cdots\supsetneq N_k
	\]
	we chosen. 
	Since $N_k$ is not minimal, there exists $N_{k+1}$ such that $N_k\supsetneq N_{k+1}$.
	This procedure produces a sequence $N_1\supsetneq
	N_2\supsetneq\cdots$ that cannot stabilize, a contradiction. 
	
	If $N_1\supseteq N_2\supseteq\cdots$ is a sequence of submodules, then 
	$\mathcal{S}=\{N_j:j\geq1\}$ has a minimal element, say $N_n$. Then
	$N_n=N_{n+k}$ for all $k$. 
\end{proof}

A modulo $N$ is \textbf{noetherian} if for every sequence 
$N_1\subseteq N_2\subseteq\cdots$ of submodules of $N$ there exists $n\in\Z_{>0}$ such that 
$N_n=N_{n+k}$ for all $k\in\Z_{>0}$. 

% Let $X$ be a set and $\mathcal{S}$ be a set of subsets of $X$. We say that 
% $B\in\mathcal{S}$ is a \textbf{maximal element} of $\mathcal{S}$ if
% there is no $Z\in\mathcal{S}$ such that $B\subsetneq Z$.

\begin{exercise}
    Let $M$ be a module. The following statements are equivalent:
    \begin{enumerate}
        \item $M$ is noetherian.
        \item Every submodule of $M$ is finitely generated. 
        \item Every non-empty subset $\mathcal{S}$ of submodules of $M$ contains a maximal element, that is
            an element $X\in\mathcal{S}$ such that there is no $Z\in\mathcal{S}$ such that $X\subseteq Z$.  
    \end{enumerate}
\end{exercise}

\begin{exercise}
\label{xca:AN_exact}
	Let 
	\[
	\begin{tikzcd}
		0 \arrow{r}
		& A \arrow{r}{f}
		& B \arrow{r}{g}
		& C \arrow{r}
		& 0
	\end{tikzcd}
	\]
	be an exact sequence of modules. Prove that $B$ is noetherian (resp.
	artinian) if and only if $A$ and $C$ are noetherian (resp. artinian).
\end{exercise}

% \begin{definition}
% 	Un anillo $R$ se dice \textbf{noetheriano a izquierda} si el módulo 
% 	$\prescript{}{R}R$ es noetheriano.
% \end{definition}
%Similarly one defines right noetherian rings.

\begin{definition}
	A ring $R$ is \textbf{left artinian} if the module 
	$\prescript{}{R}R$ is artinian.
\end{definition}

Similarly one defines right artinian rings. 

\begin{example}
	The ring $\Z$ is noetherian. It is not artinian, as the sequence
	\[
	2\Z\supseteq
	4\Z\supseteq 8\Z\supseteq\cdots
	\]
	does not stabilize. 
\end{example}

\begin{definition}
	\label{def:serie_de_composicion}
	A \textbf{composition series} of the module $M$ is a sequence 
	\[
		\{0\}=M_0\subsetneq M_1\subsetneq M_2\subsetneq\cdots\subsetneq M_n=M
	\]
	of submodules of $M$ such that each $M_i/M_{i-1}$ is non-zero and has no proper submodules. 
	In this case 
	$n$ is the length of $M$ and $M$ is said to have \textbf{finite length}.
\end{definition}

The previous definition makes sense also for non-unitary rings. That is why
it is required that each quotient $M_i/M_{i-1}$ has no proper submodules.

\begin{theorem}
	\label{thm:serie_de_composicion}
	A non-zero module admits a composition series if and only if it is artinian and noetherian.
\end{theorem}

\begin{proof}
	Let $M$ be a non-zero module and let $\{0\}=M_0\subsetneq
	M_1\subsetneq\cdots\subsetneq M_n=M$ be a composition series for $M$.
	We claim that each $M_i$ is artinian and noetherian. We proceed by induction on $i$. The case
	$i=0$ is trivial. Let us assume that $M_i$ is artinian and noetherian. Since 
	$M_i/M_{i+1}$ has no proper submodules and the sequence 
	\[
	\begin{tikzcd}
		0 \arrow{r}
		& M_i \arrow{r}
		& M_{i+1} \arrow{r}
		& M_{i+1}/M_i \arrow{r}
		& 0
	\end{tikzcd}
	\]
	is exact, it follows that 
	$M_{i+1}$ is artinian and noetherian, see Exercise \ref{xca:AN_exact}. 

    Conversely, let $M$ be an artinian and noetherian module. Let $M_0=\{0\}$ and 
    $M_1$ be minimal among the submodules of $M$ (it exists by Proposition \ref{pro:artinian_minimal}.
    If $M_1\ne M$, let 
	$M_2$ be minimal among those submodules of $M$ such that $M_1\subsetneq M_2$. This procedure
	produces a sequence 
	\[
		\{0\}=M_0\subsetneq M_1\subsetneq M_2\subsetneq\cdots
	\]
	of submodules of $M$, where each $M_{i+1}/M_i$ is non-zero and admits no
	proper submodules. Since $M$ is noetherian, the sequence stabilizes and
	hence it follows that $M_n=M$ for some $n$. 
\end{proof}

\begin{definition}
    Let $M$ be a module. 
	We say that the composition series
	\[
	M=V_0\supseteq V_1\supseteq\cdots\supseteq V_k=\{0\},
	\quad
	M=W_0\supseteq W_1\supseteq\cdots\supseteq W_l=\{0\},
	\]
	are \textbf{equivalent} if $k=l$ and there exists 
	$\sigma\in\Sym_n$ such that 
	$V_{i}/V_{i-1}\simeq W_{\sigma(i)}/W_{\sigma(i)-1}$
	for all $i\in\{1,\dots,k\}$.
\end{definition}

\begin{theorem}[Jordan--H\"older]
	\label{thm:JordanHolder}
	Any two composition series for a module are equivalent. 
\end{theorem}

\begin{proof}
    Let $M$ be a module and
    \[
		M=V_0\supseteq V_1\supseteq\cdots\supseteq V_k=\{0\},
		\quad
		M=W_0\supseteq W_1\supseteq\cdots\supseteq W_l=\{0\},
	\]
	be composition series of $M$. 
	We claim that these composition series are equivalent. 
	We proceed by induction on $k$. The case $k=1$ is trivial, as 
	in this case $M$ has no proper submodules and $M\supseteq\{0\}$ 
	is the only possible composition series for $M$. So
	assume the result holds for modules with composition series of length $<k$. If $V_1=W_1$, then 
	$V_1$ has composition series of lengths $k-1$ and $l-1$. The inductive hypothesis implies that 
	$k=l$ and we are done. So assume that $V_1\ne W_1$. Since $V_1$ and $W_1$ are submodules of $M$, the
	sum $V_1+W_1$ is also a submodule of $M$. Moreover, $V/V_1$ has no non-zero proper submodules
	and hence 
	$V_1+W_1=V$. Then 
	\[
		V/V_1=\frac{V_1+W_1}{V_1}\simeq\frac{V_1}{V_1\cap W_1}.
	\]
	Since $V_1$ has a composition series, $V_1$ is artinian and
	noetherian by Theorem~\ref{thm:serie_de_composicion}. The submodule $U=V_1\cap W_1$ is also 
	artinian and noetherian and hence, by Theorem \ref{thm:serie_de_composicion}, 
	it admits a composition series 
	\[
		U=U_0\supseteq U_1\supseteq\cdots\supseteq U_r=\{0\}.
	\]
    Thus
    $V_1\supseteq\cdots\supseteq V_k=\{0\}$ and  
	$V_1\supseteq U\supseteq U_1\supseteq\cdots\supseteq U_r=\{0\}$ are both composition 
	series for $V_1$. The inductive hypothesis implies that 
	$k-1=r+1$ and that these composition series are equivalent. Similarly, 
	\[
		W_1\supseteq W_1\supseteq\cdots\supseteq W_l=\{0\},
		\quad
		W_1\supseteq U\supseteq U_1\supseteq\cdots\supseteq U_{r}=\{0\},
	\]
    are both composition series for $W_1$ and hence $l-1=r+1$ and these composition 
    series are equivalent. Therefore $l=k$ and the proof is completed. 
\end{proof}

Jordan--H\"older's theorem allows us to define the 
length of modules that admit a composition series. 

\begin{definition}
    Let $M$ be a module with a composition series. 
    The \textbf{length} $\ell(M)$ of $M$ is defined as the length of any composition series of $M$. 
\end{definition}

A module is said to be of finite length if it admits a composition series. 

\begin{exercise}
	If $N$ and $Q$ are modules with composition series and  
	\[
	\begin{tikzcd}
		0 \arrow[r]
		& N \arrow{r}{f}
		& M \arrow{r}{g}
		& Q \arrow[r]
		& 0
	\end{tikzcd}
	\]
	is an exact sequence of modules, then $\ell(M)=\ell(N)+\ell(Q)$.
\end{exercise}

%\begin{proof}
%	Sean $Q=Q_0\supsetneq Q_1\supsetneq\cdots\supsetneq Q_m=0$ y
%	$N=N_0\supsetneq N_1\supseteq\cdots\supsetneq N_n=0$ series de composición
%	para $Q$ y $N$ respectivamente. Entonces
%	\[
%		M=g^{-1}(Q_0)\supsetneq g^{-1}(Q_1)\supsetneq\cdots\supsetneq g^{-1}(Q_m)=f(N_0)\supsetneq f(N_1)\supsetneq\cdots\supsetneq f(N_n)=0
%	\]
%	es una serie de composición para $M$ y luego $c(M)=c(N)+c(Q)$.
%\end{proof}

\begin{exercise}
	If $A$ and $B$ are finite-length submodules of $M$, then  
	\[
	\ell(A+B)+\ell(A\cap B)=\ell(A)+\ell(B).
	\]
\end{exercise}

\section*{\S6. Semisimple modules}

In the first lectures we studied semisimple modules over finite-dimensional 
algebras. Let us now review the theory of semisimple modules over rings. 
A (finitely generated) module $M$ (over a ring $R$) is \textbf{semisimple} 
if it isomorphic to a (finite) direct sum of simple modules. 

\begin{definition}
    Let $R$ be a ring. A left ideal $L$ is said to be \textbf{minimal}
    if $L\ne\{0\}$ and there is no left ideal $J$
    such that $\{0\}\subsetneq J\subsetneq I$.
\end{definition}

The ring $\Z$ contains no minimal left ideals. If $I$ is a non-zero 
left ideal of $\Z$, then
$I=(n)$ for some $n>0$ and $I=(n)\supsetneq (2n)$. 

\begin{proposition}
    Let $R$ be a left artinian ring. 
    Then every non-zero left ideal contains a minimal left ideal. 
\end{proposition}

\begin{proof}
    Let $X$ be the family of non-zero left ideals contained in $I$. Then $X$ is non-empty, as 
    $I\in X$. Then $X$ contains a minimal element by Proposition \ref{pro:artinian_minimal}. 
\end{proof}

% \begin{proposition}
% 	Let $R$ be a unitary ring and $M$ be a unitary semisimple module. 
% 	The following statements are equivalent:
% 	\begin{enumerate}
% 		\item $M$ is noetherian.
% 		\item $M$ is artinian.
% 		\item $M$ is a direct sum of finitetely many simple modules. 
% 	\end{enumerate}
% \end{proposition}

% \begin{proof}
% 	We first prove that $3)\Longleftrightarrow1)$ and that 
% 	$3)\Longleftrightarrow2)$. Como cada submódulo simple es artiniano y
% 	noetheriano, $M$ resulta artiniano y noetheriano. Recíprocamente, si $M$ es
% 	artiniano, $I$ debe ser finito pues de lo contrario podríamos elegir
% 	elementos $i_1,i_2,i_3,\dots$ de $I$ tales que la sucesión
% 	\[
% 		\bigoplus_{i\in I}M_i\supsetneq \bigoplus_{i\in I\setminus\{i_1\}}M_i\supsetneq\bigoplus_{i\in I\setminus\{i_1,i_2\}}M_i\supsetneq\cdots
% 	\]
% 	nunca se estabiliza. Análogamente, si $M$ es noetheriano, podríamos elegir
% 	elementos $i_1,i_2,i_3,\dots\in I$ tales que la sucesión
% 	\[
% 		M_{i_1}\subsetneq M_{i_1}\oplus M_{i_2}\subsetneq\cdots
% 	\]
% 	nunca se estabiliza.
% \end{proof}

A ring $R$ with identity is \textbf{semisimple} if it is a direct sum of finitely many minimal left ideals. Note
that $\prescript{}{R}{R}$ is finitely generated by $\{1\}$. Minimal left ideals of $R$ 
are exactly the simple submodules of $\prescript{}{R}{R}$. 
This means that 
the ring $R$ is semisimple if and only if the module
$\prescript{}{R}{R}$ is semisimple.  

\begin{proposition}
    Let $R$ be a semisimple ring. Then $R$ is noetherian and artinian.
\end{proposition}

\begin{proof}
    Write $R$ as a direct sum $R=L_1\oplus\cdots\oplus L_n$ of minimal left ideals. Since 
    each $L_j$ is a simple submodule of $\prescript{}{R}{R}$, it follows that 
    \[
    L_1\oplus\cdots\oplus L_n\supsetneq L_2\oplus\cdots\oplus L_n\supsetneq\cdots\supsetneq L_n\supsetneq\{0\}
    \]
    is a composition series for $\prescript{}{R}{R}$ with composition factors
    $L_1,\dots,L_n$. Since $\prescript{}{R}{R}$ admits a composition
    series, it is artinian and noetherian by Theorem \ref{thm:serie_de_composicion}.
\end{proof}

Now it is possible to prove Artin--Wedderburn's theorem for rings. 
If $R$ is a semisimple ring, then
\[
R\simeq \prod_{i=1}^k M_{n_i}(D_i)
\]
for some $n_1,\dots,n_k\geq1$ and some
division rings $D_1,\dots,D_k$. 
The proof is somewhat
the same we did for finite-dimensional algebras.

\begin{theorem}
	\label{thm:SSartin=J}
	Let $R$ be a unitary ring. Then $R$ is semisimple if and only if 
	$R$ is left artinian and $J(R)=\{0\}$.
\end{theorem}

We shall need a lemma.

\begin{lemma}
	\label{lem:Jartiniano}
	Let $R$ be a unitary left artinian ring. There exists finitely many maximal ideals 
	$I_1,\dots,I_n$ of $R$ such that 
	$J(R)=I_1\cap\cdots\cap I_n$.
\end{lemma}

\begin{proof}
	Since $R$ is
	unitary, $J(R)$ is the intersection of all maximal ideals of $R$. Since $R$ is left artinian,
	Proposition~\ref{pro:artinian_minimal} implies that 
	the set of ideals of the form
	$I_1\cap\cdots\cap I_n$ for finitely many maximal ideals $I_1,\dots,I_n$ of $R$ 
	contains a minimal element, say 
	$J=\bigcap_{i=1}^k I_i$. We claim that $J=J(R)$. If not, let $x\in
	J(R)\setminus J$. Then there exists a maximal ideal $M$ such that $x\not\in
	M$. This implies that $J\cap M\subsetneq J$, a contradiction to the minimality of 
    $J$. 
\end{proof}

We now prove the theorem. 

\begin{proof}[Proof of Theorem \ref{thm:SSartin=J}]
	Assume first that $R$ is semisimple. By Artin--Wedderburn's theorem, 
	\[
		R\simeq\prod_{i=1}^kM_{n_i}(D_i)
	\]
	for some $n_1,\dots,n_k\geq1$ and some division rings $D_1,\dots,D_k$. 
	In particular, $R$ is left artinian and $J(R)=\prod_{i=1}^kJ(M_{n_i}(D_i))=\{0\}$
	because each $M_{n_i}(D_i)$ is simple. 

    Conversely, the previous lemma implies that $\{0\}=J(R)=I_1\cap\cdots\cap I_k$ for some
    maximal ideals $I_1,\dots,I_k$. Since each $R/I_i$ is simple, it follows that 
    $\prod_{i=1}^k R/M_i$ is semisimple. Since $I_1\cap\cdots\cap I_k=\{0\}$, the map 
	$R\to \prod_{i=1}^k R/M_i$ is an injective ring homomorphism. Thus $R$ 
	is semisimple. 
\end{proof}

% Como consecuencia tenemos el siguiente resultado:

% \begin{proposition}
% 	Sea $G$ un grupo. Entonces $\C[G]$ es artiniana a izquierda si y sólo si
% 	$G$ es finito. 
% \end{proposition}

% \begin{proof}
% 	Si $G$ es finito sabemos que $\C[G]$ es artiniano a izquierda por ser de
% 	dimensión finita.  Recíprocamente, si $G$ es infinito, sabemos que
% 	$J(\C[G])=0$ (por el teorema de Rickart) y que $\C[G]$ no es semisimple
% 	(por la proposición~\ref{pro:nunca_SS}). Luego $\C[G]$ no es artiniana a izquierda por el
% 	teorema~\ref{thm:SSartin=J}.
% \end{proof}

% Concluimos la sección con el siguiente teorema:

We now present an important result that uses 
semisimplicity. 

\begin{theorem}[Hopkins--Levitszki]
	\label{thm:Hopkins-Levitski}
	Let $R$ be a unitary left artinian ring. Then $R$ is left noetherian.
\end{theorem}

\begin{proof}
	Let $J=J(R)$. Since $R$ is left artinian, $J$ is a nilpotent ideal 
	by Theorem~\ref{thm:Jnilpotente}. Let $n$ be such that $J^n=0$. Now consider the sequence 
	\[
		R\supsetneq J\supsetneq J^2\supsetneq\cdots\supsetneq J^{n-1}\supsetneq J^n=\{0\}.
	\]
	Each $J^{i}/J^{i+1}$ is a module over $R$ annihilated by $J$, so each  
	$J^i/J^{i+1}$ is a module over $(R/J)$. Since $R/J$ is left artinian and 
	$J(R)=\{0\}$, it follows from the previous proposition that $R/J$ is semisimple. 
	It follows that each $J^{i}/J^{i+1}$ 
	is semisimple and hence it is left noetheriano. Inductively one proves that each 
	$J^i$ is left noetherian and therefore $R$ is left noetherian. 
\end{proof}


\section*{\S7. Rickart's theorem}

Let $K$ be a field and $G$ be a group. The \textbf{group algebra} $K[G]$ 
is the vector space (over $K$) with basis $\{g:g\in G\}$ 
and the algebra structure given by the multiplication
\[
	\left(\sum_{g\in G}\lambda_gg\right)\left(\sum_{h\in G}\mu_hh\right)
	=\sum_{g,h\in G}\lambda_g\mu_h(gh).
\]
Note that every element of $K[G]$ is a finite sum of the form $\sum_{g\in G}\lambda_gg$.

\begin{exercise}
\label{xc:K[G]notsimple}
    If $G$ is non-trivial, then $K[G]$ is not simple. 
\end{exercise}

\begin{exercise}
	Let $G=C_n$ be the (multiplicative) cyclic group of order $n$. Prove that 
	$K[G]\simeq K[X]/(X^n-1)$. 
\end{exercise}

\begin{exercise}
	Let $G$ be a finitely-generated torsion-free abelian group. Prove that 
	$K[G]$ is a domain. 
\end{exercise}

\begin{exercise}
	Let $G$ be a group and $H$ be a subgroup of $G$. Let $\alpha\in K[H]$. Prove that 
    $\alpha$ is invertible (resp. left zero divisor) in $K[H]$ if and only if 
	$\alpha$ is invertible (resp. left zero divisor) in
	$K[G]$.
\end{exercise}

\begin{exercise}
	Let $G$ be a group and $\alpha=\sum_{g\in G}\lambda_gg\in K[G]$.  
	The \textbf{support} of $\alpha$ is the set 
	\[
		\supp\alpha=\{g\in G:\lambda_g\ne 0\}.
	\]
	Prove that if $g\in G$, then 
	$\supp(g\alpha)=g(\supp\alpha)$ and $\supp(\alpha g)=(\supp\alpha)g$.
\end{exercise}

% El objetivo de esta sección es calcular el radical de Jacobson del álgebra de
% grupo de un grupo finito. Comenzamos con un ejemplo:

\begin{exercise}
	Let $G=C_2=\langle g\rangle\simeq\Z/2$ the (multiplicative) 
	group with two elements. Note that every element of $K[G]$ is of the form
	$a1+bg$ for some $a,b\in K$. Prove the following statements:
	\begin{enumerate}
	    \item If the characteristic of $K$ is different from two, then 
	    \[
		K[G]\to K\times K,
		\quad
		a1+bg\mapsto (a+b,a-b),
	\]
	is an algebra isomorhism. 
	\item If the characteristic of $K$ is two, then 
	\[
	K[G]\to \begin{pmatrix}
			K & K\\
			0 & K
		\end{pmatrix},
		\quad
		a1+bg\mapsto\begin{pmatrix}
			a+b & b\\
			0 & a+b
		\end{pmatrix},
	\]
	is an algebra isomorphism. 
	\end{enumerate}
\end{exercise}

Veamos otros ejemplo un poco más difíciles. La idea a utilizar es la siguiente:
Si $A$ es una $K$-álgebra y $\rho\colon G\to U(A)$ es un morfismo de grupos,
donde $U(A)$ es el grupo de unidades de $A$, entonces la función $K[G]\to A$,
$\sum_{g\in G}\lambda_gg\mapsto\sum_{g\in G}\lambda_g\rho(g)$, es un morfismo
de álgebras.

\begin{exercise}
	Let $G=C_3$ be the (multiplicative) group of three elements. Prove that
	$\R[G]\simeq\R\times\C$.
% 	Escribamos $G=\langle g:g^3=1\rangle$ y sea 
% 	\[
% 		\varphi\colon\R[G]\to\R\times\C,
% 		\quad
% 		g\mapsto (1,\omega),
% 	\]
% 	donde $\omega$ es una raíz cúbica primitiva de la unidad. Entonces
% 	$\varphi$ es inyectivo pues
% 	$0=\varphi(a1+bg+cg^2)=(a+b+c,a+b\omega+c\omega^2)$ implica que $a=b=c=0$.
% 	Luego $\varphi$ es un isomorfismo pues
% 	$\dim_\R\R[G]=\dim_\R(\R\times\C)=3$. 
\end{exercise}

\begin{exercise}
	Let $G=\langle r,s:r^3=s^2=1,\,srs=r^{-1}\rangle$ be the dihedral group of six elements. 
	Prove the following statements:
	\begin{enumerate}
	    \item $\C[G]\simeq\C\times\C\times M_2(\C)$.
	    \item $\Q[G]\simeq\Q\times\Q\times M_2(\Q)$.
	\end{enumerate}  
% 	Sea $\omega$ una raíz cúbica de la unidad y sean  
% 	\[
% 		R=\begin{pmatrix}
% 			\omega & 0\\
% 			0 & \omega^2
% 		\end{pmatrix},
% 		\quad
% 		S=\begin{pmatrix}
% 			0 & 1\\
% 			1 & 0
% 		\end{pmatrix}.
% 	\]
% 	Un cálculo sencillo muestra que $R^2=S^2=I$ y que $SRS=R^{-1}$. Sea
% 	\[
% 		\varphi\colon\C[G]\to\C\times\C\times M_2(\C),\quad
% 		r\mapsto (1,1,R),\quad
% 		s\mapsto (1,-1,S).
% 	\]
% 	Es fácil ver que $\varphi$ es un morfismo de álgebras. Veamos que es
% 	biyectivo. Como $\dim_{\C}\C[G]=\dim_{\C}(\C\times\C\times M_2(\C))=6$,
% 	basta ver que $\varphi$ es inyectivo. Si 
% 	\[
% 		\alpha=a_0+a_1r+a_2r^2+(b_0+b_1r+b_2r^2)s\in\ker\varphi,
% 	\]
% 	entonces 
% 	\[
% 		0=\varphi(\alpha)=\left(u,v,\begin{pmatrix} \alpha_{11} & \alpha_{12}\\\alpha_{21}&\alpha_{22}\end{pmatrix}\right), 
% 	\]
% 	donde
% 	\begin{align*}
% 		&u = a_0+a_1+a_2+b_0+b_1+b_2, && v = a_0+a_1+a_2-b_0-b_1-b_2,\\
% 		&\alpha_{11}=a_0+a_1\omega+a_2\omega^2, && \alpha_{12}=b_0+b_1\omega+b_2\omega^2,\\
% 		&\alpha_{21}=b_0+b_2\omega+b_1\omega^2, && \alpha_{22}=a_0+a_2\omega+a_1\omega^2.
% 	\end{align*}
% 	Un cálculo sencillo muestra que estas ecuaciones implican que
% 	$\alpha=0$ y luego $\varphi$ es inyectiva.  
\end{exercise}

We now consider the following problem. 

\begin{openproblem}
Let $G$ be a group and $K$ be a field. When $J(K[G])=\{0\}$?
\end{openproblem}

As an application of Amitsur's theorem we prove that 
complex group algebras have null Jacobson radical.
This is known as 
Rickart's theorem. The original proof found by Rickart 
uses complex analysis. Here, however, 
we present an algebraic proof. 


\begin{theorem}[Rickart]
\label{thm:J(C[G])=0}
    Let $G$ be a group. Then $J(\C[G])=\{0\}$.
\end{theorem}

To prove the theorem we need a lemma.

\begin{lemma}
Let $G$ be a group. Then $J(\C[G])$ is nil.        
\end{lemma}

\begin{proof}
    We need to show that every element of $J(\C[G])$ is nilpotent. 
    If $G$ is countable, then the result follows from Amitsur's theorem. So assume that 
    $G$ is not countable. Let $\alpha\in J(\C[G])$, say
    \[
    \alpha=\sum_{i=1}^n\lambda_ig_i,
    \]
    where $\lambda_1,\dots,\lambda_n\in\C$ and $g_1,\dots,g_n\in G$. Let $H=\langle g_1,\dots,g_n\rangle$.
    Then $g\in \C[H]$ and $H$ is countable. We claim that $g\in J(\C[H])$. Decompose
    $G$ as a disjoint union 
    \[
    G=\bigcup_\lambda x_\lambda H
    \]
    of cosets of $H$ in $G$. Then $\C[G]=\bigoplus_\lambda x_\lambda\C[H]$ and
    hence $\C[G]=\C[H]\oplus K$ for some right module $K$ over $\C[H]$. Since $\alpha\in J(\C[G])$, for each 
    $\beta\in\C[H]$ there exists $\gamma\in\C[G]$ such that 
    $\gamma(1-\beta\alpha)=1$. Write $\gamma=\gamma_1+\kappa$ for $\gamma_1\in\C[H]$ and $\kappa\in K$. Then
    \[
    1=\gamma(1-\beta\alpha)=\gamma_1(1-\beta\alpha)+\kappa(1-\beta\alpha)
    \]
    and hence $\kappa(1-\beta\alpha)\in K\cap \C[H]=\{0\}$. Since $1=\gamma_1(1-\beta\alpha)$, it follows that
    $\alpha\in J(\C[H])$ and the lemma follows from Amitsur's theorem.  
\end{proof}

We now prove the theorem. 

\begin{proof}[Proof of Theorem \ref{thm:J(C[G])=0}]
    For $\alpha=\sum_{i=1}^n\lambda_ig_i\in\C[G]$ let 
    \[
    \alpha^*=\sum_{i=1}^n\overline{\lambda_i}g_i^{-1}.
    \]
    Then $\alpha\alpha^*=0$ if and only if $\alpha=0$ and, moreover, 
    $(\alpha\beta)^*=\beta^*\alpha^*$ for all $\beta\in\C[G]$. 
    Assume that $J(\C[G])\ne\{0\}$ and let $\alpha\in J(\C[G])\setminus\{0\}$. Then
    $\beta=\alpha\alpha^*\in J(\C[G])$, as $J(\C[G])$ is an ideal of $\C[G]$. Moreover, $\beta\ne 0$, as 
    \[
    (\beta^m)^*=(\beta^*)^m=\beta^m
    \]
    for all $m\geq1$. If there exists $k\geq2$ such that $\beta^k=0$ and $\beta^{k-1}\ne 0$, then
    \[
    \beta^{k-1}\left(\beta^{k-1}\right)^*=\beta^{2k-2}=0
    \]
    and hence $\beta^{k-1}=0$, a contradiction. Thus $\beta=0$ and therefore $\alpha=0$. 
\end{proof}

To obtain a consequence of Rickart's theorem we need two lemmas. 

\begin{lemma}[Nakayama]
	\label{lem:Nakayama}
	Let $R$ be a unitary ring and $M$ be a finitely generated module. If 
	$J(R)\cdot M=M$, then $M=\{0\}$.
\end{lemma}

\begin{proof}
    Since $M$ is finitely generated, we may assume that 
	$M=(x_1,\dots,x_n)$. Since $x_n\in M=J(R)\cdot M$, 
	there exist $r_1,\dots,r_n\in J(R)$ such that $x_n=r_1\cdot x_1+\cdots+r_n\cdot x_n$, that is 
	$(1-r_n)\cdot x_n=\sum_{j=1}^{n-1}r_j\cdot x_j$. 
	Since $1-r_n$ is invertible, there exists $s\in R$ such that $s(1-r_n)=1$. Thus 
	$x_n=\sum_{j=1}^{n-1}(sr_j)\cdot x_j$ 
	and hence $M=(x_1,\dots,x_{n-1})$. Repeating this procedure several times 
	one obtains $M=\{0\}$.
\end{proof}

\begin{lemma}
	\label{lem:Rickart}
	Let $\iota\colon R\to S$ be a homomorphism of unitary rings. If	
	\[
	S=\iota(R)x_1+\cdots+\iota(R)x_n,
	\]
	where each $x_j$ is such that $x_jy=yx_j$ for all $y\in\iota(R)$, then 
	$\iota(J(R))\subseteq J(S)$.
\end{lemma}

\begin{proof}
	We claim that $J=\iota(J(R))$ acts trivially on each simple $S$-module $M$.
	If is $M$ is a simple module over $S$, then, in particular, $M=S\cdot m$ for some $m\ne0$. 
	Now $M$ is a module over $R$ with $r\cdot m=\iota(r)\cdot m$. Since 
	\[
		M=S\cdot m=(\iota(R)x_1+\cdots+\iota(R)x_n)\cdot m=\iota(R)\cdot (x_1\cdot m)+\cdots+\iota(R)\cdot (x_n\cdot m),
	\]
	it follows that 
	$M$ is finitely generated as a module over $\iota(R)$. Moreover, 
	\[
	J(R)\cdot
	M=J\cdot M=\iota(J)\cdot M
	\]
	is an $S$-submodule of $M$, as 
	\[
		x_j\cdot (J\cdot M)=(x_j J)\cdot M=(J x_j)\cdot M=J\cdot (x_j\cdot M)\subseteq J\cdot M.
	\]
	Since $M\ne\{0\}$, Nakayama's lemma implies that $J(R)\cdot M\subsetneq M$. The simplicity of 
	the $S$-module $M$ implies that $J(R)\cdot M=\{0\}$.
\end{proof}

We now obtain the following consequence of Rickart's theorem. 

\begin{theorem}
	If $G$ is a group, then $J(\R[G])=0$. 
\end{theorem}

\begin{proof}
	Let $\iota\colon \R[G]\to\C[G]$ be the canonical inclusion. Since 
	\[
	\C[G]=\R[G]+i\R[G],
	\]
	Lemma~\ref{lem:Rickart} and Rickart's theorem imply that 
	$\iota(J(\R[G]))\subseteq J(\C[G])=0$. Thus $J(\R[G])=0$, as $\iota$ is injective. 
\end{proof}

We now characterize when complex group algebras 
are left artinian. For that purpose
we need a lemma.

\begin{lemma}
    Let $M$ be a semisimple module and $N$ be a submodule. 
    Then $N$ is a direct summand.
\end{lemma}

\begin{proof}[Sketch of the proof]
    Let $M=\oplus_{i\in I}M_i$ be a direct sum of simple modules  
    and let $i\in I$. 
    Since $N\cap M_i$ is a submodule of $M_i$ and $M_i$ is simple, it follows
    that $N\cap M_i=\{0\}$ or $N\cap M_i=M_i$. If
    $N\cap M_i=M_i$ for all $i\in I$, then $N=M$ and the lemma is proved. So we may assume
    that there exists $i\in I$ such that $N\cap M_i=\{0\}$. Let $X$ be the set
    of subsets $J$ of $I$ such that $N\cap (\oplus_{j\in J}M_j)=\{0\}$. Our assumptions
    imply that $X$ is non-empty. Zorn's lemma implies the existence of 
    a maximal element $K$. Let $N_1=\oplus_{k\in K}M_k$. We claim that
    $N\oplus N_1=M$. If not, there exists $i\in I$ such that
    $M_i\not\subseteq N\oplus N_1$. The simplicity of $M_i$ implies that
    $M_i\cap (N\cap N_1)=\{0\}$, which contradicts the maximality of $K$. 
\end{proof}

A direct application of the lemma proves that
complex group algebras of infinite groups are never semisimple. 

\begin{proposition}
    If $G$ is an infinite group, then $\C[G]$ is not semisimple. 
\end{proposition}

\begin{proof}
    % Consider the proper non-zero ideal 
    % \[
	   % I(G)=\left\{\sum_{g\in G}\lambda_gg\in\C[G]:\sum_{g\in G}\lambda_g=0\right\}.
    % \]
    % Note that $\dim I(G)=\dim\C[G]-1$, so $I$ is a proper non-zero ideal of $\C[G]$.
\end{proof}

The ideal $I(G)$ used in the proof of the previous proposition 
is known as the augmentation ideal
of the complex group algebra $\C[G]$ of $G$. 

\begin{theorem}
	Let $G$ be a group. Then $\C[G]$ 
	is left artinian if and only if 
	$G$ is finite. 
\end{theorem}

\begin{proof}
    If $G$ is finite, then $\C[G]$ is left artinian because $\dim\C[G]=|G|<\infty$. So assume that 
    $G$ is infinite. By Rickart's theorem,   
	$J(\C[G])=0$. Moreover, $\C[G]$
	is not semisimple by the previous proposition. Thus
	$\C[G]$ is not left artinian by Theorem~\ref{thm:SSartin=J}.
\end{proof}

\section*{\S8. Maschke's theorem}

We now present another instance of the Jacobson semisimplicity problem.
In this case, our result is for finite groups. 

\begin{theorem}[Maschke]
	Let $G$ be a finite group. Then $J(K[G])=0$ if and only 
	if the characteristic of $K$ is zero 
	or does not divide the order of $G$. 
\end{theorem}

\begin{proof}
	Supongamos que $G=\{g_1,\dots,g_n\}$ con $g_1=1$. Sea $\rho\colon K[G]\to
	K$ dada por $\alpha\mapsto\trace(L_{\alpha})$, donde
	$L_{\alpha}(\beta)=\alpha\beta$. Tenemos $\rho(g_1)=n$ y $\rho(g_i)=0$ para
	todo $i\in\{2,\dots,n\}$ pues,  como $L_{g_i}(g_j)=g_{i}g_j\ne g_j$, la
	matriz de $L_{g_i}$ en la base $\{g_1,\dots,g_n\}$ tiene ceros en la
	diagonal.

	Supongamos que $J=J(K[G])$ es no nulo y sea
	$\alpha=\sum_{i=1}^n\lambda_ig_i\in J\setminus\{0\}$. Sin pérdida de
	generalidad podemos suponer que $\lambda_1\ne 0$ (pues si $\lambda_1=0$ hay
	algún $\lambda_i\ne 0$ y alcanza con tomar $g_i^{-1}\alpha\in J$). Entonces
	\[
		\rho(\alpha)=\sum_{i=1}^n \lambda_i\rho(g_i)=n\lambda_1.
	\]
	Como $G$ es un grupo finito, $K[G]$ es un álgebra de dimensión finita y
	luego $K[G]$ es artiniana a izquierda. Como el radical de Jacobson $J$ es
	un ideal nilpotente, en particular $\alpha$ es un elemento nil. Luego
	$L_{\alpha}$ es nilpotente y entonces $0=\rho(\alpha)=n\lambda_1$. Esto
	implica que la característica del cuerpo $K$ divide a $n$. 

	Recíprocamente, supongamos que la característica de $K$ es un número primo
	que divide a $n$ y sea $\alpha=\sum_{i=1}^ng_i$. Como $\alpha
	g_j=g_j\alpha=\alpha$ para todo $j\in\{1,\dots,n\}$, el conjunto
	$I=K[G]\alpha$ es un ideal de $K[G]$. Como además 
	\[
		\alpha^2=\sum_{i=1}^n g_i\alpha=n\alpha=0,
	\]
	se concluye que $I$ es un ideal no nulo y nilpotente. Luego $J(K[G])\ne 0$
	pues por la proposición~\ref{pro:nilJ} sabemos que $I\subseteq J(K[G])$.
\end{proof}

\begin{corollary}
	\label{cor:GfinitoNOnil}
	Sea $G$ un grupo finito. Entonces $K[G]$ no contiene ideales a izquierda
	nil no nulos.
\end{corollary}

\begin{proof}
	Es consecuencia inmediata del teorema de Maschke ya que $J(K[G])$ contiene a
	todo ideal a izquierda nil.	
\end{proof}

%\index{Anillo!semisimple}
%Recordemos que un anillo unitario $R$ se dice \textbf{semisimple} si para cada
%ideal $I$ de $R$ existe un ideal $J$ de $R$ tal que $R=I\oplus J$.
%
%%\begin{corollary}
%%	Sea $G$ un grupo finito y $K$ un cuerpo de característica coprima con el
%%	orden de $G$. Entonces $K[G]$ es semisimple.
%%\end{corollary}
%%
%%\begin{proof}
%%	
%%\end{proof}
%
%\begin{theorem}
%	Si $G$ es un grupo infinito, entonces $K[G]$ nunca es semisimple.
%\end{theorem}
%
%\begin{proof}
%	Sea $R=K[G]$ y supongamos que $R$ es semisimple.  Si $I$ es el ideal de
%	aumentación de $R$, existe un ideal no nulo $J$ de $R$ tal que $R=I\oplus
%	J$. Como $R$ es unitario, existen $e\in I$, $f\in J$ tales que $1=e+f$. Si
%	$x\in I$, entonces $x=xe+xf$ y luego $xf=x-xe\in I\cap J=\{0\}$. Como
%	entonces $x=xe$ para todo $x\in I$, en particular $e_1=e_1^2$. Análogamente
%	vemos que $e_2^2=e_2$. Además $ef=0$ pues $ef\in I\cap J=\{0\}$.  Como $I$
%	es el ideal de aumentación y $If=(Re)f=R(ef)=0$, se concluye que $(g-1)f=0$
%	para todo $g\in G$ pues $g-1\in I$. Si suponemos que $f=\sum_{h\in
%	G}\lambda_hh$, entonces 
%	\[
%	f=gf=\sum_{h\in G}\lambda_h(gh)=\sum_{h\in
%	G}\lambda_{g^{-1}h}h.
%	\]
%	Luego $\lambda_h=\lambda_{g^{-1}h}$ para todo $g,h\in G$, una contradicción
%	pues como $f\ne 0$ la suma que define a $f$ es infinita. 
%\end{proof}

