\chapter{}

\begin{definition}
	\index{Módulo!artiniano}
	Sea $R$ un anillo. 
	Un $R$-módulo $N$ se dice \textbf{artiniano} si para toda sucesión
	$N_1\supseteq N_2\supseteq\cdots$ de submódulos de $N$ existe $n\in\N$ tal
	que $N_n=N_{n+k}$ para todo $k\in\N$.
\end{definition}

\begin{definition}
	\index{Módulo!noetheriano}
	Sea $R$ un anillo. 
	Un $R$-módulo $N$ se dice \textbf{noetheriano} si para toda sucesión
	$N_1\subseteq N_2\subseteq\cdots$ de submódulos de $N$ existe $n\in\N$ tal
	que $N_n=N_{n+k}$ para todo $k\in\N$.
\end{definition}

\index{Minimal!elemento}
\index{Maximal!elemento}
Sea $X$ un conjunto y sea $\mathcal{S}$ un conjunto de subconjuntos de $X$.
Diremos que $A\in\mathcal{S}$ es un \textbf{elemento minimal} en $\mathcal{S}$
si no existe $Y\in\mathcal{S}$ tal que $Y\subsetneq A$. Similarmente, 
que $B\in\mathcal{S}$ es un \textbf{elemento maximal} en $\mathcal{S}$ si no
existe $Z\in\mathcal{S}$ tal que $B\subsetneq Z$.

\begin{lemma}
	\label{lem:modulo_artiniano}
	Un módulo $N$ es artiniano si y sólo si todo conjunto no
	vacío de submódulos de $N$ tiene un elemento minimal.
\end{lemma}

\begin{proof}
	Supongamos que $N$ es artiniano y que . Sea $\mathcal{S}$ el conjunto (no
	vacío) de submódulos de $N$. Supongamos que $\mathcal{S}$ no tiene elemento
	minimal y sea $N_1\in\mathcal{S}$.  Como $N_1$ no es minimal, existe
	$N_2\in\mathcal{S}$ tal que $N_1\supsetneq N_2$.  Supongamos que tenemos
	elegidos $k$ submódulos $N_1\supsetneq N_2\supseteq\cdots\supsetneq N_k$.
	Como $N_k$ no es minimal, existe $N_{k+1}$ tal que $N_k\supsetneq N_{k+1}$.
	De esta forma tenemos una sucesión de submódulos $N_1\supsetneq
	N_2\supsetneq\cdots$ que no se estabiliza, una contradicción.

	Recíprocamente, si $N_1\supseteq N_2\supseteq\cdots$ es una sucesión de
	submódulos, entonces el conjunto $\mathcal{S}=\{N_j:j\geq1\}$ tiene un
	elemento minimal $N_n$. Luego $N_n=N_{n+k}$ para todo $k\in\N$.
\end{proof}

\begin{lemma}
	\label{lem:noetheriano1}
	Un módulo $N$ es noetheriano si y sólo si todo conjunto no vacío de
	submódulos de $N$ tiene un elemento maximal.
\end{lemma}

\begin{proof}
	Es similar a la prueba del lema~\ref{lem:modulo_artiniano}.
\end{proof}

\begin{lemma}
	\label{lem:noetheriano2}
	Un módulo $N$ es noetheriano si y sólo si todo submódulo de $N$ es
	finitamente generado.
\end{lemma}

\begin{proof}
	Supongamos que $N$ es noetheriano. Sea $P$ un submódulo de $N$ y sea
	$\mathcal{S}$ el conjunto de submódulos de $P$ finitamente generados. Como
	$0\in\mathcal{S}$, el conjunto $\mathcal{S}$ es no vacío. Por el
	lema~\ref{lem:noetheriano1}, existe $B\in\mathcal{S}$ elemento maximal.
	Sea $\{b_1,\dots,b_m\}$ un conjunto finito de generadores de $B$. Si $p\in
	P$ entonces, por maximalidad, $B$ está contenido en el submódulo de $P$
	generado por $\{p,b_1,\dots,b_m\}$. Como este submódulo está en
	$\mathcal{S}$, se concluye que $p\in B$.

	Sea $N_1\subseteq N_2\subseteq\cdots$ una subcesión de submódulos de $N$.
	Como $M=\cup_{j\geq1}N_j$ es un submódulo de $N$, es finitamente generado.
	Sea $\{x_1,\dots,x_m\}$ un conjunto de generadores de $M$. Como cada $x_j$
	está en algún $N_k$, existe $n\in\N$ tal que $M\subseteq N_n$. En
	particular, $N_{n}=N_{n+k}$ para todo $k\geq1$. 
\end{proof}

\begin{exercise}
	\label{exercise:noetheriano:exact}
	Sea 
	\[
	\begin{tikzcd}
		0 \arrow{r}
		& A \arrow{r}{f}
		& B \arrow{r}{g}
		& C \arrow{r}
		& 0
	\end{tikzcd}
	\]
	una sucesión exacta de $R$-módulos. Demuestre que $B$ es noetheriano (resp.
	artiniano) si y sólo si $A$ y $C$ son noetherianos (resp. artinianos).
\end{exercise}

%Como consecuencia %del ejercicio~\ref{exercise:exact} obtenemos los siguientes
%\framebox{FIXME}
%resultados:
%
%\begin{exercise}
%	\label{exercise:noetheriano:B/A}
%	Sea $B$ un submódulo de $A$. Demuestre que $B$ es noetheriano (resp.
%	artiniano) si y sólo si $A$ y $B/A$ son noetherianos (resp. artinianos).
%\end{exercise}
%
%\begin{exercise}
%	\label{exercise:noetheriano:direct}
%	Sean $A_1,\dots,A_n$ $R$-módulos. Demuestre que $A_1\oplus\cdots\oplus A_n$
%	es noetheriano (resp. artiniano) si y sólo si cada $A_i$ es noetheriano
%	(resp. artiniano).
%\end{exercise}


\begin{definition}
	\index{Anillo!noetheriano a izquierda}
	\index{Anillo!noetheriano a derecha}
	Un anillo $R$ se dice \textbf{noetheriano a izquierda} si el módulo 
	$\prescript{}{R}R$ es noetheriano.
\end{definition}

Análogamente se definen anillos noetherianos a derecha.

\begin{definition}
	\index{Anillo!artiniano a izquierda}
	\index{Anillo!artiniano a derecha}
	Un anillo $R$ se dice \textbf{artiniano a izquierda} si el módulo
	$\prescript{}{R}R$ es artiniano.
\end{definition}

Análogamente se definen anillos artinianos a derecha.

\begin{example}
	Del lema~\ref{lem:noetheriano2} se obtiene inmediatamente que $\Z$ es
	noetheriano. Sin embargo, $\Z$ no es noetheriano pues $2\Z\supseteq
	4\Z\supseteq 8\Z\supseteq\cdots$ es una sucesión de ideales que no se
	estabiliza.
\end{example}

\begin{definition}
	\label{def:serie_de_composicion}
	\index{Módulo!serie de composición de un}
	\index{Módulo!de longitud finita}
	Una \textbf{serie de composición} para un módulo $M$ es una sucesión
	\[
		0=M_0\subsetneq M_1\subsetneq M_2\subsetneq\cdots\subsetneq M_n=M
	\]
	de submódulos de $M$ tal que cada $M_i/M_{i-1}$ es no nulo y no tiene
	submódulos propios. En este caso, $n$ es la longitud de $M$ y $M$ se dice 
	de \textbf{longitud finita}.
\end{definition}

La definición~\ref{def:serie_de_composicion} tiene sentido también en anillos
no unitarios, y por eso pedimos que cada cociente $M_i/M_{i-1}$ no tenga
submódulos propios (esto garantiza la simplicidad en el caso de anillos
unitarios).

\begin{theorem}
	\label{thm:serie_de_composicion}
	Un módulo no nulo admite una serie de composición si y sólo si es artiniano
	y noetheriano.
\end{theorem}

\begin{proof}
	Sea $M$ un módulo no nulo y sea $0=M_0\subsetneq
	M_1\subsetneq\cdots\subsetneq M_n=M$ una serie de composición para $M$.
	Demostraremos por inducción que cada $M_i$ es artiniano y noetheriano. El
	caso $i=0$ es trivial. Supongamos entonces que $M_i$ es artinoano y
	noetheriano. Como $M_i/M_{i+1}$ no tiene submódulos propios y la sucesión 
	\[
	\begin{tikzcd}
		0 \arrow{r}
		& M_i \arrow{r}
		& M_{i+1} \arrow{r}
		& M_{i+1}/M_i \arrow{r}
		& 0
	\end{tikzcd}
	\]
	es exacta, se concluye que $M_{i+1}$ es artiniano y noetheriano. 

	Supongamos ahora que $M$ es un módulo noetheriano y artiniano. Sea $M_0=0$
	y sea $M_1$ minimal entre los submódulos de $M$ (existe $M_1$ gracias al
	lema~\ref{lem:modulo_artiniano} pues $M$ es artiniano). Si $M_1\ne M$, sea
	$M_2$ minimal entre los submódulos de $M$ tales que $M_1\subsetneq M_2$. Al
	continuar de esta forma obtenemos una sucesión
	\[
		0=M_0\subsetneq M_1\subsetneq M_2\subsetneq\cdots
	\]
	de submódulos de $M$, donde cada $M_{i+1}/M_i$ es no nulo y no admite
	submódulos propios. Como $M$ es noetheriano, la sucesión se estabiliza y
	luego $M_n=M$ para algún $n$. 
\end{proof}

\begin{definition}
	Diremos que las series de composición 
	\[
	V=V_0\supseteq V_1\supseteq\cdots\supseteq V_k=0,
	\quad
	V=W_0\supseteq W_1\supseteq\cdots\supseteq W_l=0,
	\]
	son \textbf{equivalentes} si $k=l$ y además existe una permutación
	$\sigma\in\Sym_n$ tal que para todo $i\in\{1,\dots,k\}$ se tiene
	$V_{i}/V_{i-1}\simeq W_{\sigma(i)}/W_{\sigma(i)-1}$.
\end{definition}

\begin{theorem}[Jordan--H\"older]
	\label{thm:JordanHolder}
	\index{Teorema!de Jordan--H\"older}
	\index{Serie de composición}
	Sea $V$ un módulo y sean 
	\[
		V=V_0\supseteq V_1\supseteq\cdots\supseteq V_k=0,
		\quad
		V=W_0\supseteq W_1\supseteq\cdots\supseteq W_l=0,
	\]
	dos series de composición para $V$. Entonces las series son
	equivalentes.
\end{theorem}

\begin{proof}
	Procederemos por inducción en $k$. El caso $k=1$ es trivial pues en este
	caso $V$ no tiene submódulos propios y luego $V\supseteq 0$ es la única
	serie de composición. Supongamos entonces que el resultado vale para
	módulos con series de composición de longitud $<k$.  Si $V_1=W_1$ entonces
	$V_1$ tiene dos series de composición de longitudes $k-1$ y $l-1$. Por
	hipótesis inductiva, $k=l$ y el teorema queda demostrado.  Supongamos
	entonces que $V_1\ne W_1$. Como $V_1$ y $W_1$ son submódulos de $V$, la
	suma $V_1+W_1$ es un submódulo de $V$. Además, como $V/V_1$ no tiene
	submódulos propios no nulos, $V_1+W_1=V$. Sea $U=V_1\cap W_1$. Entonces
	\[
		V/V_1=\frac{V_1+W_1}{V_1}\simeq\frac{V_1}{V_1\cap W_1}.
	\]
	Como $V_1$ tiene una serie de composición, $V_1$ es artiniano y
	noetheriano por el teorema~\ref{thm:serie_de_composicion}. Entonces
	también $U$ es artiniano y noetheriano, y luego tiene una serie de
	composición por el teorema~\ref{thm:serie_de_composicion}, digamos
	\[
		U=U_0\supseteq U_1\supseteq\cdots\supseteq U_r=0.
	\]
	Tenemos entonces que 
	$V_1\supseteq\cdots\supseteq V_k=0$ y 
	$V_1\supseteq U\supseteq U_1\supseteq\cdots\supseteq U_r=0$ 
	son series de composición para $V_1$. Por hipótesis inductiva,
	$k-1=r+1$ y las series de composición son equivalentes. Similarmente, 
	\[
		W_1\supseteq W_1\supseteq\cdots\supseteq W_l=0,
		\quad
		W_1\supseteq U\supseteq U_1\supseteq\cdots\supseteq U_{r}=0,
	\]
	son series de composición para $W_1$ y entonces $l-1=r+1$ y las series son
	equivalentes. Luego $l=k$ y el teorema queda demostrado.
\end{proof}

\begin{definition}
\index{Módulo!longitud}
Si $M$ es un módulo que admite una serie de composición, digamos
$0=M_0\subsetneq M_1\subsetneq\cdots\subsetneq M_n=M$, 
se define la \textbf{longitud} de $M$ como el entero $c(M)=n$.
\end{definition}

\begin{exercise}
	Si $N$ y $Q$ son módulos de longitud finita y 
	\[
	\begin{tikzcd}
		0 \arrow[r]
		& N \arrow{r}{f}
		& M \arrow{r}{g}
		& Q \arrow[r]
		& 0
	\end{tikzcd}
	\]
	es una sucesión exacta de módulos, entonces $c(M)=c(N)+C(Q)$.
\end{exercise}

%\begin{proof}
%	Sean $Q=Q_0\supsetneq Q_1\supsetneq\cdots\supsetneq Q_m=0$ y
%	$N=N_0\supsetneq N_1\supseteq\cdots\supsetneq N_n=0$ series de composición
%	para $Q$ y $N$ respectivamente. Entonces
%	\[
%		M=g^{-1}(Q_0)\supsetneq g^{-1}(Q_1)\supsetneq\cdots\supsetneq g^{-1}(Q_m)=f(N_0)\supsetneq f(N_1)\supsetneq\cdots\supsetneq f(N_n)=0
%	\]
%	es una serie de composición para $M$ y luego $c(M)=c(N)+c(Q)$.
%\end{proof}

\begin{exercise}[teorema de la dimensión]
	Sean $A$ y $B$ submódulos de $M$ de longitud finita. Demuestre que entonces 
	$c(A+B)+c(A\cap B)=c(A)+C(B)$.
\end{exercise}

\begin{lemma}
	Sea $R$ un anillo unitario y $M$ un módulo unitario semisimple. Son equivalentes:
	\begin{enumerate}
		\item $M$ es noetheriano.
		\item $M$ es artiniano.
		\item $M$ es suma directa de finitos simples.
	\end{enumerate}
\end{lemma}

\begin{proof}
	Veamos ahora que $(3)\Longleftrightarrow(1)$ y que
	$(3)\Longleftrightarrow(2)$. Como cada submódulo simple es artiniano y
	noetheriano, $M$ resulta artiniano y noetheriano. Recíprocamente, si $M$ es
	artiniano, $I$ debe ser finito pues de lo contrario podríamos elegir
	elementos $i_1,i_2,i_3,\dots$ de $I$ tales que la sucesión
	\[
		\bigoplus_{i\in I}M_i\supsetneq \bigoplus_{i\in I\setminus\{i_1\}}M_i\supsetneq\bigoplus_{i\in I\setminus\{i_1,i_2\}}M_i\supsetneq\cdots
	\]
	nunca se estabiliza. Análogamente, si $M$ es noetheriano, podríamos elegir
	elementos $i_1,i_2,i_3,\dots\in I$ tales que la sucesión
	\[
		M_{i_1}\subsetneq M_{i_1}\oplus M_{i_2}\subsetneq\cdots
	\]
	nunca se estabiliza.
\end{proof}

Veremos ahora la relación existente entre la semisimplicidad y el radical de
Jacobson. Primero necesitamos un lema:

\begin{lemma}
	\label{lem:Jartiniano}
	Sea $R$ un anillo unitario y artiniano a izquierda. Entonces existen
	finitos ideales maximales $I_1,\dots,I_n$ de $R$ tales que
	$J(R)=I_1\cap\cdots\cap I_n$.
\end{lemma}

\begin{proof}
	Como $R$ es
	unitario, $J(R)$ es la intersección de ideales maximales de $R$.  Como $R$
	es artiniano a izquierda, el lema~\ref{lem:modulo_artiniano} nos dice que
	el conjunto de ideales formados por la intersección de finitos ideales
	maximales $I_1,\dots,I_k$ de $R$ posee un elemento minimal, digamos
	$J=\bigcap_{i=1}^k I_i$. Veamos que $J=J(R)$. Si no, sea $x\in
	J(R)\setminus J$.  Existe entonces un ideal maximal $M$ tal que $x\not\in
	M$. Pero entonces $J\cap M\subsetneq J$, una contradicción a la minimalidad
	de $J$. 
\end{proof}

\begin{theorem}
	\label{thm:SSartin=J}
	Si $R$ es un anillo unitario, entonces $R$ es semisimple si y sólo si $R$
	es artiniano a izquierda y $J(R)=0$.
\end{theorem}

\begin{proof}
	Supongamos primero que $R$ es semisimple. Por el teorema de Wedderburn,
	existen enteros positivos $n_1,\dots,n_k$ y anillos de división
	$D_1,\dots,D_k$ tales que \[
		R\simeq\prod_{i=1}^kM_{n_i}(D_i).
	\]
	En particular, $R$ es
	artiniano a izquierda y $J(R)=\prod_{i=1}^kJ(M_{n_i}(D_i))=0$ pues
	cada $M_{n_i}(D_i)$ es simple. 

	Recíprocamente, por el lema anterior sabemos que $0=J(R)=I_1\cap\cdots\cap I_k$ para
	finitos ideales maximales $I_1,\dots,I_k$.  Como cada cociente $R/I_i$ es
	simple, $\prod_{i=1}^k R/M_i$ es semisimple. Como $I_1\cap\cdots\cap I_k=0$,
	el morfismo $R\to \prod_{i=1}^k R/M_i$ es inyectivo y luego $R$ es también
	semisimple.
\end{proof}

Como consecuencia tenemos el siguiente resultado:

\begin{proposition}
	Sea $G$ un grupo. Entonces $\C[G]$ es artiniana a izquierda si y sólo si
	$G$ es finito. 
\end{proposition}

\begin{proof}
	Si $G$ es finito sabemos que $\C[G]$ es artiniano a izquierda por ser de
	dimensión finita.  Recíprocamente, si $G$ es infinito, sabemos que
	$J(\C[G])=0$ (por el teorema de Rickart) y que $\C[G]$ no es semisimple
	(por la proposición~\ref{pro:nunca_SS}). Luego $\C[G]$ no es artiniana a izquierda por el
	teorema~\ref{thm:SSartin=J}.
\end{proof}

Concluimos la sección con el siguiente teorema:

\begin{theorem}[Hopkins--Levitszki]
	\label{thm:Hopkins-Levitski}
	Si $R$ es un anillo unitario artiniano a izquierda, entonces $R$ es
	noetheriano a izquierda.
\end{theorem}

\begin{proof}
	Sea $J=J(R)$. Como $R$ es artiniano a izquierda, $J$ es un ideal nilpotente
	por el teorema~\ref{thm:Jnilpotente}, digamos $J^n=0$. Consideremos la
	sucesión
	\[
		R\supsetneq J\supsetneq J^2\supsetneq\cdots\supsetneq J^{n-1}\supsetneq J^n=0.
	\]
	Cada $J^{i}/J^{i+1}$ es un $R$-módulo anulado por $J$. Luego cada 
	$J^i/J^{i+1}$ es un $(R/J)$-módulo. Como $R/J$ es artiniano (pues $R$ lo
	es) y $J(R)=0$, $R/J$ es semisimple. Luego cada $J^{i}/J^{i+1}$ es
	semisimple y entonces es noetheriano a izquierda.  Inductivamente se
	demuestra entonces que cada $J^i$ es noetheriano a izquierda y luego $R$
	también lo es.
\end{proof}

%\section{Anillos semiprimitivos y semiprimos}

\begin{definition}
	\index{Anillo!semiprimitivo}
	\index{Anillo!semisimple Jacobson}
	Un anillo $R$ se dice \textbf{semiprimitivo} (o semisimple Jacobson) si
	$J(R)=0$.
\end{definition}

\begin{example}
	Si $R$ es primitivo entonces es semiprimitivo. En efecto, como $R$ es
	primitivo, $\{0\}$ es un ideal primitivo y luego, como $J(R)$ es la
	intersección de los ideales primitivos de $R$, se concluye que $J(R)=0$.
\end{example}

\begin{example}
	Si $R=\prod_{i\in I}R_i$ es producto directo de anillos semiprimitivos,
	entonces $R$ es semiprimitivo pues 
	\[
		J(R)=J\left(\prod_{i\in I}R_i\right)=J\left(\prod_{i\in I}J(R_i)\right)=0.
	\]
\end{example}

\begin{example}
	$\Z$ es semiprimitivo pues $J(\Z)=\cap_{p}\Z/p=\{0\}$.
\end{example}

\begin{example}
	Sea $R=C[a,b]$ el anillo de funciones $f\colon [a,b]\to\R$ continuas. Como
	$R$ es un anillo unitario, $J(R)$ es la intersección de los ideales
	maximales de $R$. Todo ideal maximal de $R$ es de la forma
	\[
		U_c=\{f\in C[a,b]:f(c)=0\}
	\]
	para algún $c\in[a,b]$. En efecto, es fácil ver que cada $U_c$ es un ideal;
	$U_c$ es maximal pues $C[a,b]/U_c\simeq\R$.  Luego $J(R)=\cap_{a\leq c\leq
	b}U_c=0$.
\end{example}

\begin{theorem}
	\label{thm:semiprimitivo}
	Si $R$ es un anillo, entonces $R/J(R)$ es semiprimitivo. 
\end{theorem}

\begin{proof}
	Si $R$ es un anillo radical, el resultado es trivial. Supongamos entonces
	que $J(R)\ne R$ y sea $M$ un módulo simple. Entonces $M$ es un
	$R/J(R)$-módulo simple con
	\[
		(x+J(R))m=xm,\quad
		x\in R,\,m\in M.
	\]
	Si $x+J(R)\in J(R/J(R))$ entonces $xM=(x+J(R))M=0$. Luego $x\in J(R)$ pues
	$x$ anula a cualquier módulo simple de $R$.
\end{proof}

%El teorema de densidad de Jacobson nos permite entonces obtener el siguiente resultado:
%
%\begin{theorem}
%	Sea $R$ un anillo no radical. Entonces $R/J(R)$ es isomorfo a un producto
%	subdirecto de anillos densos en espacios vectoriales sobre anillos de
%	división.	
%\end{theorem}
%
%\begin{proof}
%	Si $R$ no es radical, $J(R)\ne R$. Luego $R/J(R))$ es semiprimitivo por el
%	teorema~\ref{thm:semiprimitivo}. El teorema~\ref{thm:subdirecto} y el
%	teorema de densidad de Jacobson completan la demostración del teorema.
%\end{proof}


\begin{definition}
	\index{Producto subdirecto de anillos}
	Sea $\{R_i:i\in I\}$ una familia de anillos. Un subanillo $R$ de
	$\prod_{i\in I}R_i$ se dice un \textbf{producto subdirecto} de los $R_j$ si
	cada $\pi_j\colon R\to R_j$ es sobreyectiva. 
\end{definition}

El siguiente teorema justifica que indistintamente llamemos anillos
semiprimitivos a los anillos semisimples Jacobson:

\begin{theorem}
	\label{thm:subdirecto}
	Sea $R$ un anillo no nulo. Entonces $R$ semiprimitivo si y sólo si $R$ es
	isomorfo a un producto subdirecto de anillos primitivos.
\end{theorem}

\begin{proof}
	Supongamos que $R$ es semiprimitivo y sea $\{P_i:i\in I\}$ la familia de
	ideales primitivos de $R$. Cada $R/P_j$ es primitivo y
	$\{0\}=J(R)=\cap_{i\in I}P_i$. Para cada $j$, sean $\lambda_j\colon R\to
	R/P_j$ y $\pi_j\colon \prod_{i\in I}R/P_i\to R/P_j$ los morfismos
	canónicos. La función
	\[
		\phi\colon R\to\prod_{i\in I}R/P_i,\quad
		r\mapsto \{\lambda_i(r):i\in I\},
	\]
	es un morfismo inyectivo de anillos tal que $\pi_j\phi(R)=R/P_j$ para todo
	$j$.

	Supongamos ahora que $R$ es isomorfo a un producto subrirecto de anillos
	$R_j$ primitivos y sea $\varphi\colon R\to\prod_{i\in I}R_i$ un morfismo
	inyectivo tal que $\pi_j(\varphi(R))=R_j$ para todo $j$. Para cada $j$ sea
	$P_j=\ker\pi_j\varphi$. Como $R/P_j\simeq R_j$, cada $P_j$ es un ideal
	primitivo. Si $x\in\cap_{i\in I}P_i$ entonces $\varphi(x)=0$ y luego $x=0$.
	Luego $J(R)\subseteq\cap_{i\in I} P_i=0$. 
\end{proof}

\begin{example}
	El anillo $\Z$ es isomorfo a un producto subdirecto de los cuerpos $\Z/p$
	con $p$ primo.
\end{example}

\begin{example}
	El anillo $C[a,b]$ es isomorfo a un producto subdirecto de los cuerpos
	$C[a,b]/U_c\simeq\R$.
\end{example}

\begin{definition}
	Un anillo $R$ se dice \textbf{semiprimo} si para todo $a\in R$ tal que
	$aRa=0$ se tiene que $a=0$.
\end{definition}

\begin{lemma}
	Sea $R$ un anillo. Son equivalentes:
	\begin{enumerate}
		\item $R$ es semiprimo.
		\item Si $I$ es un ideal a izquierda tal que $I^2=0$ entonces $I=0$.
		\item Si $I$ es un ideal tal que $I^2=0$ entonces $I=0$.
		\item $R$ no tiene ideales nilpotentes no nulos. 
	\end{enumerate}
\end{lemma}

\begin{proof}
	Veamos que $(1)\implies(2)$. Si $I^2=0$ y $x\in I$, entonces $xRx\subseteq I^2=0$ y
	luego $x=0$. Las implicaciones $(2)\implies(3)$ y $(4)\implies(3)$ son triviales. Veamos que
	$(3)\implies(4)$.  Si $I$ es un ideal nilpotente no nulo, sea $n\in\N$
	minimal tal que $I^n=0$.  Como $(I^{n-1})^2=0$, $I^{n-1}=0$, una
	contradicción. Por último veamos que $(3)\implies(1)$. Sea $a\in R$ tal que
	$aRa=0$. Entonces $I=RaR$ es un ideal de $R$ tal que $I^2=0$. Por hipótesis, $RaR=I=0$. Luego
	$Ra$ y $aR$ son ideales tales que $(Ra)R=R(aR)=0$. Esto implica que $\Z a$ es un ideal de $R$
	tal que $(\Z a)R=0$ y luego $a=0$.
\end{proof}

\begin{example}
	Un anillo conmutativo es semiprimo si y sólo si no tiene elementos
	nilpotentes no nulos.
\end{example}


\begin{proposition}
	El anillo $\C[G]$ es semiprimo.
\end{proposition}

\begin{proof}
	Como $J(\C[G])=0$ por el teorema de Rickart y además el radical de Jacboson
	contiene a todo ideal nil por la proposición~\ref{pro:nilJ}, se deduce que
	$\C[G]$ no tiene ideales nil no triviales. Tampoco tiene entonces ideales
	nilpotentes no triviales y luego $\C[G]$ es semiprimo.
\end{proof}

\begin{exercise}
	Demuestre que $Z(\C[G])$ es semiprimo.
\end{exercise}

% tomar $\alpha$ tal que $\alpha^2=0$ y sea $A=K[G]\alpha$. Como $A^2=0$, $A=0$ y entonces $\alpha=0$.

\begin{example}
	Sea $D$ un anillo de división. Entonces $D[X]$ es semiprimo.
\end{example}

\begin{example}
	Sea $D$ un anillo de división. Entonces $D[[X]]$ es semiprimo y no es
	semiprimitivo.
\end{example}




%\section{Anillos semiprimitivos}
%
%\begin{lemma}
%	\label{lem:Iunitario}
%	Sea $R$ un anillo y sea $I$ un ideal de $R$ unitario. Sea $e\in I$ la
%	unidad de $I$. Entonces $e$ es un idempotente central de $R$, $I=eR$ y
%	existe un ideal $J$ de $R$ tal que $R=I\oplus J$. Además $R\simeq I\times
%	J$.
%\end{lemma}
%
%\begin{proof}
%	Como $e\in I$, $eR\subseteq I$. Luego $I=eR$ pues $I=eI\subseteq eR$. Como
%	$ex\in I$ y $xe\in I$ para todo $x\in R$, $ex=(ex)e$ y $xe=e(xe)$. Luego
%	$ex=xe$ y entonces $e$ es central e idempotente. Sea $J=\{x-ex:x\in R\}$.
%	Es fácil demostrar que $J$ es un ideal tal que $R=I\oplus J$. Además
%	$R\simeq I\times J$, via $x\mapsto (ex,x-ex)$,
%\end{proof}
%
%A continuación daremos una demostración muy sencilla del teorema de Wedderburn
%en el caso de álgebras de dimensión finita.
%
%\begin{theorem}[Artin--Wedderburn]
%	Sea $R$ un anillo artiniano a izquierda y no nulo. Entonces $R$ es
%	semiprimo si y sólo si existen $n_1,\dots,n_r\in\N$ y existen anillos de
%	división $D_1,\dots,D_r$ tales que $R\simeq M_{n_1}(D_1)\times\cdots\times
%	M_{n_r}(D_r)$.
%\end{theorem}
%
%\begin{proof}
%	Procederemos por inducción en $\dim A$. Si $\dim A=1$\dots\framebox{} 
%
%	Supongamos entonces que $\dim A>1$. Si $A$ es un álgebra prima, el
%	resultado se sigue inmediatamente del teorema de Wedderburn. Supongamos
%	entonces que existe $a\in A\setminus\{0\}$ tal que $I=\{x\in A:aAx=0\}$ es
%	no nulo. Como $I$ es un ideal de $A$, $I$ es un álgebra semiprima.
%	\framebox{?} Como $a\not\in I$, $\dim I<\dim A$, y entonces, por hipótesis
%	inductiva, existen $n_1,\dots,n_s\in\N$ y álgebras de división
%	$D_1,\dots,D_s$ tales que 
%	\[
%		I\simeq M_{n_1}(D_1)\times\cdots\times M_{n_s}(D_s).
%	\]
%	En particular, $I$ es unitario. Por el lema~\ref{lem:Iunitario}, existe un
%	ideal $J$ de $A$ tal que $A\simeq I\times J$. Como $\dim J<\dim A$, la hipótesis inductiva
%	implica que existen $n_{s+1},\dots,n_r\in\N$ y álgebras de división $D_{s+1},\dots,D_r$ tales que
%	\[
%		J\simeq M_{n_{s+1}}(D_{s+1})\times\cdots\times M_{n_r}(D_r).
%	\]
%	Luego $A\simeq I\times J\simeq \prod_{j=1}^s M_{n_j}(D_j)$.
%\end{proof}
%
%\begin{corollary}
%	Sea $A$ un álgebra no nula de dimensión finita. Si $A$ es semiprima,
%	entonces $A$ es unitaria.
%\end{corollary}
%
%%Gracias al teorema de Wedderburn se puede ir un poco más lejos:
%%\begin{corollary}
%%	Sea $A$ un álgebra unitaria. Son equivalentes:
%%	\begin{enumerate}
%%		\item $A$ es semiprima.
%%		\item Todo $A$-módulo unitario es semisimple.
%%		\item $A$ es semisimple como $A$-módulo.
%%		\item Todo ideal a izquierda de $A$ es de la forma $Ae$ para algún
%%			idempotente $e\in A$. 
%%	\end{enumerate}
%%\end{corollary}
%%
%%\begin{proof}
%%	La implicación $(1)\implies(2)$ es el teorema de Wedderburn. 
%%	
%%\end{proof}
%
%\begin{example}
%	Por el teorema de Maschke sabemos que si $G$ es un grupo finito, 
%	$\C[G]$ es un álgebra semiprimitiva y luego semisimple.
%\end{example}
%




%\section{Viejo!}
%
%\begin{theorem}[Artin--Wedderburn]
%	\index{Teorema!de Artin--Wedderburn}
%	\label{thm:ArtinWedderburn}
%	Si $R$ es un anillo, las siguientes afirmaciones son equivalentes:
%	\begin{enumerate}
%		\item $R$ es un anillo no nulo semiprimitivo y artiniano a izquierda.
%		\item Existen anillos de división $D_1,\dots,D_r$ y tales que
%			\[
%				R\simeq\prod_{i=1}^r R_i,
%			\]
%			donde $R_i=\End_{D_i}(V_i)$
%		\item Existen anillos de división $D_1,\dots,D_r$ y enteros positivos
%			$n_1,\dots,n_r$ tales que 
%			\[
%			R\simeq M_{n_1}(D_1)\times\cdots\times M_{n_r}(D_r).
%		\]
%	\end{enumerate}
%\end{theorem}
%
%\begin{proof}
%	Demostremos que $(1)\implies(2)$. Como $R\ne0$ y $J(R)=0$, $R$ admite
%	ideales primitivos. Supongamos que existe un número finito de ideales
%	primitivos distintos, digamos $P_1,\dots,P_t$. Cada $R/P_j$ es un anillo
%	primitivo y es artiniano a izquierda \framebox{?}. Entonces, por el teorema
%	de Wedderburn, para cada $j\in\{1,\dots,t\}$ existen un anillo de división
%	$D_j$ y un entero positivo $n_j$ tales que $R/P_j\simeq M_{n_j}(D_j)$. En
%	particular, cada $R/P_j$ es simple y entonces $P_j$ es un ideal maximal de
%	$R$. Como $R/P_j$ es simple, $(R/P_j)^2\ne 0$ y luego $R^2\not\subseteq
%	P_j$. Por maximalidad, $R^2+P_j=R$ y además $P_i+P_j=R$ para todo $i\ne j$.
%	Por el teorema chino del resto,
%	\[
%		R=R/0=R/J(R)=R/\cap_{j=1}^t P_j\simeq R/P_1\times\cdots\times R/P_t.
%	\]
%	Sea $\iota_k\colon R/P_k\to \prod_{j=1}^t R/P_j$ la inclusión canónica.
%	Cada $\iota_k(R/P_k)$ es un ideal simple (es decir, que como anillo es
%	simple) de $\prod_{j=1}^t R/P_j\simeq R$. Luego las imágenes, digamos
%	$I_k$, de los $\iota_k(R/P_k)$ dan ideales simples de $R$ y
%	$R=I_1\times\cdots\times I_t$.
%
%	Demostremos ahora que $(3)\implies(1)$. Para cada $j$ sea
%	$R_j=M_{n_j}(D_j)$. Como cada $R_j$ es primitivo por el teorema de
%	Wedderburn, $J(R_j)=\{0\}$ para todo $j$. Luego
%	$J(R)=\prod_{i=1}^rJ(R_j)=\{0\}$ y entonces $R$ es semiprimitivo. Además
%	$R$ es artiniano a izquierda.\framebox{?}
%\end{proof}
%
%\begin{corollary}
%	Sea $R$ un anillo semiprimitivo.
%	\begin{enumerate}
%		\item Si $R$ es artiniano a izquierda, entonces $R$ es unitario.
%		\item $R$ es artiniano a izquierda si y sólo si es artiniano a derecha.
%		\item Si $R$ es artiniano a izquierda es noetheriano.
%	\end{enumerate}
%\end{corollary}
%
%\begin{proof}
%	La primera afirmación es consecuencia inmediata del teorema de
%	Artin--Wedderburn~\ref{thm:ArtinWedderburn}.
%\end{proof}
%
%\begin{corollary}
%	Sea $R$ un anillo semiprimitivo artiniano a izquierda y sea $I$ un ideal de
%	$R$. Entonces $I=Re$ para algún idempotente $e\in R$ tal que $e\in Z(R)$.
%\end{corollary}
%
%\begin{proof}
%		
%\end{proof}
%
%\begin{proposition}
%	Sea $R$ un anillo semisimple artiniano a izquierda. 
%	\begin{enumerate}
%		\item $R=I_1\times\cdots\times I_n$ donde los $I_j$ son ideales simples.
%		\item Si $J\subseteq R$ es un ideal simple, entonces existe $k\in\{1,\dots,n\}$ tal que $J=I_k$.
%		\item Si $R=J_1\times\cdots\times J_m$ donde los $J_k$ son ideales simples, entonces $n=m$ y existe
%			$\sigma\in\Sym_n$ tal que $I_k=J_{\sigma(k)}$ para todo $k\in\{1,\dots,n\}$.
%	\end{enumerate}
%\end{proposition}
%
%\begin{proof}
%\end{proof}
%
%\begin{theorem}
%	Sea $R$ un anillo unitario no nulo. Las siguientes afirmaciones son
%	equivalentes:
%	\begin{enumerate}
%		\item $R$ es semiprimitivo y artiniano a izquierda.
%		\item Todo $R$-módulo unitario es proyectivo.
%		\item Todo $R$-módulo unitario es inyectivo.
%		\item Toda sucesión exacta de $R$-módulos unitarios se parte.
%		\item Todo $R$-módulo unitario no nulo es semisimple.
%		\item $\prescript{}{R}R$ es unitario y semisimple.
%		\item Todo ideal a izquierda de $R$ es de la forma $Re$ para algún $e\in R$ indempotente.
%		\item $\prescript{}{R}R$ es suma directa de ideales a izquierda
%			minimales $L_1,\dots,L_m$ donde cada $L_j$ es de la forma $Re_j$, y
%			los $e_j$ son idempotentes ortogonales tales que
%			$e_1+\cdots+e_m=1$. 
%	\end{enumerate}
%\end{theorem}
%
%\begin{proof}
%	Veamos que $(4)\implies(5)$. Sea $M$ un módulo unitario y sea $N$ un
%	submódulo no nulo de $M$. Como la sucesión $0\to N\to M\to M/N\to 0$ es
%	exacta, se parte. Luego $M=N\oplus X$ para algún submódulo $X$ de $N$ tal
%	que $X\simeq M/N$. Como $M$ es unitario, $Rm\ne 0$ para todo $m\in
%	M\setminus\{0\}$. Luego $M$ es semisimple por el teorema~\framebox{?}.
%
%	Veamos ahora que $(5)\implies(4)$. Sea 
%	\[
%	\begin{tikzcd}
%		0 \arrow[r]
%		& N \arrow[r]
%		& M \arrow[r]
%		& X \arrow[r]
%		& 0
%	\end{tikzcd}
%	\]
%	una sucesión exacta corta de $R$-módulos. Como $f\colon N\to f(N)$ es un
%	isomorfismo y entonces $f(N)$ es un submódulo del semisimple $M$, $f(N)$ es
%	sumando directo de $M$. Sea $\pi\colon M\to f(N)$ el morfismo canónico.
%	Entonces $\pi f=f$ y $f^{-1}\pi\colon M\to A$ es un morfismo tal que
%	$(f^{-1}\pi)f=\id_N$.\framebox{?}
%
%	Demostremos que $(5)\implies(7)$. Sea $L$ un ideal a izquierda de $R$. Como
%	los ideales a izquierda de $R$ son los submódulos de $\prescript{}{R}R$,
%	existe un ideal a izquierda $N$ de $R$ tal que $R=L\oplus N$. Existen
%	entonces $e_1\in L$ y $e_2\in N$ tales que $1=e_1+e_2$. Si $x\in L$,
%	entonces $x=xe_1+xe_2$ y luego $xe_2=x-xe_1\in L\cap N=\{0\}$. Demostramos
%	entonces que $x=xe_1$ para todo $x\in L$. En particular, $e_1e_1=e_1$ y
%	$L=Re_1$. 
%
%	Demostremos que $(7)\implies(6)$. Sea $L$ un submódulo de
%	$\prescript{}{R}R$. Como entonces $L$ es un ideal a izquierda de $R$,
%	$L=Re$ para algún idempotente $e\in R$. Como el conjunto $R(1-e)$ es un
%	ideal a izquierda de $R$ tal que $R=Re\oplus R(1-e)$, se concluye que
%	$\prescript{}{R}R$ es semisimple.\framebox{?}
%
%	Veamos que $(6)\implies(1)$. Supongamos que $\prescript{}{R}R=\sum_{i\in
%	I}N_i$, donde los $N_j$ son submódulos simples de $\prescript{}{R}R$.
%	Reordenando los $N_j$ si fuera necesario, podemos suponer que existe
%	$k\in\N$ tal que $1=e_1+\cdots+e_k$ y $e_j\in N_j$ para todo
%	$j\in\{1,\dots,k\}$. Si $r\in R$, entonces $r=re_1+\cdots+re_k\in
%	\sum_{i=1}^k N_i$. Luego $R=\sum_{i=1}^k N_i$.
%	Veamos que $J(R)=0$. Si $r\in J(R)$ entonces, como $rN_i=0$ para todo
%	$i\in\{1,\dots,k\}$, se concluye que $r=r1=re_1+\cdots+re_k=0$. Probamos
%	entonces que $R$ es semiprimitivo. Falta ver que $R$ es artiniano a
%	izquierda. Para eso basta obvervar que, como
%	\[
%		\frac{N_1\oplus\cdots\oplus N_i}{N_1\oplus\cdots\oplus N_{i-1}}\simeq N_i
%	\]
%	para cada $i\in\{1,\dots,k\}$, la serie
%	\[
%	R=N_1\oplus\cdots\oplus N_k\supsetneq N_1\oplus\cdots\oplus N_{k-1}\supsetneq\cdots\supsetneq N_1\oplus N_2\supsetneq N_1\supsetneq 0
%	\]
%	es una serie de composición.\framebox{?}
%
%	Veamos que $(1)\implies(8)$. Sin perder generalidad podemos suponer que
%	\[
%	R=\prod_{i=1}^k M_{n_j}(D_j),
%	\]
%	donde los $D_j$ son anillos de división.\framebox{?}
%
%	Veamos que $(8)\implies(5)$. Sea $M$ un módulo unitario no nulo. Si $m\in
%	M$ entonces $L_im$ es un submódulo de $M$. Los $L_jm$ generan a $M$ pues 
%	\[
%	m=1m=e_1m+\cdots+e_km\in\sum L_im.
%	\]
%	Veamos que cada $L_jm$ es simple. Fijado $i$, la función $f\colon L_i\to
%	L_im$, $x\mapsto xm$, es un morfismo sobreyectivo. Como $L_i$ es un ideal a
%	izquierda minimal, $L_i$ es un submódulo simple. Luego $m\ne0$ implica que
%	$f$ es un isomorfismo. Probamos entonces que el conjunto $\{L_jm:1\leq
%	j\leq k,m\in M,L_jm\ne 0\}$ es una familia de submódulos simples cuya suma
%	es $M$.
%\end{proof}

