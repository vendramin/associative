\section{Lecture: 05/11/2024}

\subsection{Rickart's theorem}

We now consider Jacobson's semisimplicity problem. 

\begin{question}
\label{Jacobson's semisimplicity problem}
Let $G$ be a group and $K$ be a field. When $J(K[G])=\{0\}$?
\end{question}

As an application of Amitsur's theorem \ref{thm:Amitsur}, 
we prove that 
complex group algebras have null Jacobson radical.
This is known as 
Rickart's theorem. The original proof found by Rickart 
uses complex analysis. Here, however, 
we present an algebraic proof. 

\begin{theorem}[Rickart]
\index{Rickart's theorem}
\label{thm:Rickart}
    Let $G$ be a group. Then $J(\C[G])=\{0\}$.
\end{theorem}

To prove the theorem, we need a lemma.

\begin{lemma}
Let $G$ be a group. Then $J(\C[G])$ is nil.        
\end{lemma}

\begin{proof}
    We need to show that every element of $J(\C[G])$ is nilpotent. 
    If $G$ is countable, then the result follows from Amitsur's theorem \ref{thm:Amitsur}. So assume that 
    $G$ is not countable. Let $\alpha\in J(\C[G])$, say
    \[
    \alpha=\sum_{i=1}^n\lambda_ig_i,
    \]
    where $\lambda_1,\dots,\lambda_n\in\C$ and $g_1,\dots,g_n\in G$. Let $H=\langle g_1,\dots,g_n\rangle$.
    Then $\alpha\in \C[H]$ and $H$ is countable. We claim that $\alpha\in J(\C[H])$. Decompose
    $G$ as a disjoint union 
    \[
    G=\bigcup_\lambda x_\lambda H
    \]
    of cosets of $H$ in $G$. Then $\C[G]=\bigoplus_\lambda x_\lambda\C[H]$ and
    hence $\C[G]=\C[H]\oplus K$ for some right module $K$ over $\C[H]$ (this follows
    from the fact that one of the cosets is that of $H$). Since $\alpha\in J(\C[G])$, for each 
    $\beta\in\C[H]$ there exists $\gamma\in\C[G]$ such that 
    $\gamma(1-\beta\alpha)=1$. Write $\gamma=\gamma_1+\kappa$ for $\gamma_1\in\C[H]$ and $\kappa\in K$. Then
    \[
    1=\gamma(1-\beta\alpha)=\gamma_1(1-\beta\alpha)+\kappa(1-\beta\alpha)
    \]
    and hence $\kappa(1-\beta\alpha)\in K\cap \C[H]=\{0\}$, as $\beta\in\mathbb{C}[H]$. 
    Since $1=\gamma_1(1-\beta\alpha)$, it follows that
    $\alpha\in J(\C[H])$ and the lemma follows from Amitsur's theorem \ref{thm:Amitsur}.  
\end{proof}

We now prove the theorem. 

\begin{proof}[Proof of Theorem \ref{thm:Rickart}]
    For $\alpha=\sum_{i=1}^n\lambda_ig_i\in\C[G]$ let 
    \[
    \alpha^*=\sum_{i=1}^n\overline{\lambda_i}g_i^{-1}.
    \]
    Then $\alpha\alpha^*=0$ if and only if $\alpha=0$ and, moreover, 
    $(\alpha\beta)^*=\beta^*\alpha^*$ for all $\beta\in\C[G]$. 
    Assume that $J(\C[G])\ne\{0\}$ and let $\alpha\in J(\C[G])\setminus\{0\}$. Then
    $\beta=\alpha\alpha^*\in J(\C[G])$, as $J(\C[G])$ is an ideal of $\C[G]$. Moreover, the previous 
    lemma implies that $\beta$ is nilpotent. Note that $\beta\ne 0$, as $\alpha\ne0$. Since $\beta^*=\beta$,  
    \[
    (\beta^m)^*=(\beta^*)^m=\beta^m
    \]
    for all $m\geq1$. If there exists $k\geq2$ such that $\beta^k=0$ and $\beta^{k-1}\ne 0$, then
    \[
    \beta^{k-1}\left(\beta^{k-1}\right)^*=\beta^{2k-2}=0
    \]
    and hence $\beta^{k-1}=0$, a contradiction. Thus $\beta=0$ and therefore $\alpha=0$. 
\end{proof}

\begin{exercise}
	If $G$ is a group, then $J(\R[G])=0$. 
\end{exercise}

% To obtain a consequence of Rickart's theorem we need two lemmas. 

% \begin{lemma}[Nakayama]
% 	\label{lem:Nakayama}
% 	\index{Nakayama's lemma}
% 	Let $R$ be a unitary ring and $M$ be a finitely generated module. If 
% 	$J(R)\cdot M=M$, then $M=\{0\}$.
% \end{lemma}

% \begin{proof}
%     Since $M$ is finitely generated, we may assume that 
% 	$M=(x_1,\dots,x_n)$. Since $x_n\in M=J(R)\cdot M$, 
% 	there exist $r_1,\dots,r_n\in J(R)$ such that $x_n=r_1\cdot x_1+\cdots+r_n\cdot x_n$, that is 
% 	$(1-r_n)\cdot x_n=\sum_{j=1}^{n-1}r_j\cdot x_j$. 
% 	Since $1-r_n$ is invertible, there exists $s\in R$ such that $s(1-r_n)=1$. Thus 
% 	$x_n=\sum_{j=1}^{n-1}(sr_j)\cdot x_j$ 
% 	and hence $M=(x_1,\dots,x_{n-1})$. Repeating this procedure several times 
% 	one obtains $M=\{0\}$.
% \end{proof}

% \begin{lemma}
% 	\label{lem:Rickart}
% 	Let $\iota\colon R\to S$ be a homomorphism of unitary rings. If	
% 	\[
% 	S=\iota(R)x_1+\cdots+\iota(R)x_n,
% 	\]
% 	where each $x_j$ is such that $x_jy=yx_j$ for all $y\in\iota(R)$, then 
% 	$\iota(J(R))\subseteq J(S)$.
% \end{lemma}

% \begin{proof}
% 	We claim that $J=\iota(J(R))$ acts trivially on each simple $S$-module $M$.
% 	If is $M$ is a simple module over $S$, then, in particular, $M=S\cdot m$ for some $m\ne0$. 
% 	Now $M$ is a module over $R$ with $r\cdot m=\iota(r)\cdot m$. Since 
% 	\[
% 		M=S\cdot m=(\iota(R)x_1+\cdots+\iota(R)x_n)\cdot m=\iota(R)\cdot (x_1\cdot m)+\cdots+\iota(R)\cdot (x_n\cdot m),
% 	\]
% 	it follows that 
% 	$M$ is finitely generated as a module over $\iota(R)$. Moreover, 
% 	\[
% 	J(R)\cdot
% 	M=J\cdot M=\iota(J)\cdot M
% 	\]
% 	is an $S$-submodule of $M$, as 
% 	\[
% 		x_j\cdot (J\cdot M)=(x_j J)\cdot M=(J x_j)\cdot M=J\cdot (x_j\cdot M)\subseteq J\cdot M.
% 	\]
% 	Since $M\ne\{0\}$, Nakayama's lemma implies that $J(R)\cdot M\subsetneq M$. The simplicity of 
% 	the $S$-module $M$ implies that $J(R)\cdot M=\{0\}$.
% \end{proof}

% We now obtain the following consequence of Rickart's theorem. 

% \begin{theorem}
% 	If $G$ is a group, then $J(\R[G])=0$. 
% \end{theorem}

% \begin{proof}
% 	Let $\iota\colon \R[G]\to\C[G]$ be the canonical inclusion. Since 
% 	\[
% 	\C[G]=\R[G]+i\R[G],
% 	\]
% 	Lemma~\ref{lem:Rickart} and Rickart's theorem imply that 
% 	$\iota(J(\R[G]))\subseteq J(\C[G])=0$. Thus $J(\R[G])=0$, as $\iota$ is injective. 
% \end{proof}


\begin{definition}
	\index{Ring!semiprime}
	A ring $R$ \emph{semiprime} if 
	$aRa=\{0\}$ implies $a=0$.
\end{definition}

\begin{proposition}
	Let $R$ be a ring. The following statements are equivalent: 
	\begin{enumerate}
		\item $R$ is semiprime.
		\item If $I$ is a left ideal such that $I^2=\{0\}$, then $I=\{0\}$.
		\item If $I$ is an ideal such that $I^2=\{0\}$, then $I=\{0\}$.
		\item $R$ does not contain non-zero nilpotent ideals.
	\end{enumerate}
\end{proposition}

\begin{proof}
	We first prove that $1)\implies2)$. If $I^2=\{0\}$ y $x\in I$, then
	$xRx\subseteq I^2=\{0\}$ and thus $x=0$. 
    
    The implications $2)\implies3)$
	and $4)\implies3)$ are both trivial. 
    
    Let us prove that $3)\implies4)$.  If
	$I$ is a non-zero nilpotent ideal, let $n\in\Z_{>0}$ be minimal such that
	$I^n=\{0\}$.  Since $(I^{n-1})^2=\{0\}$, it follows that $I^{n-1}=\{0\}$, a
	contradiction.  
    
    Finally, we prove that $3)\implies1)$. Let $a\in R$ be such
	that $aRa=\{0\}$. Then $I=RaR$ is an ideal of $R$ such that $I^2=\{0\}$. Thus 
	$RaR=\{0\}$. This means that $Ra$ and $aR$ are ideals such that
	$(Ra)R=R(aR)=\{0\}$ (for example, $R(aR)\subseteq RaR=\{0\}\subseteq aR$). 
	Moreover, since $(Ra)(Ra)=\{0\}$ and $(aR)(aR)=\{0\}$, it follows that
	$aR=Ra=\{0\}$. 
	This implies that $\Z a$ is an ideal of $R$, as $R(\Z a)\subseteq \Z(Ra)=\{0\}$ and 
	$(\Z a)R\subseteq aR=\{0\}$. Now $(\Z a)(\Z a)\subseteq (\Z a)R=\{0\}$ and hence
	$a=0$, as $\Z a=\{0\}$. 
\end{proof}

Two consequences:

\begin{exercise}
	A commutative ring is semiprime if and only if it does not contain non-zero
	nilpotent elements. 
\end{exercise}

% tomar $\alpha$ tal que $\alpha^2=0$ y sea $A=K[G]\alpha$. Como $A^2=0$, $A=0$ y entonces $\alpha=0$.

\begin{exercise}
\label{xca:D_semiprime_semiprimitive}
	Let $D$ be a division ring. 
	\begin{enumerate}
		\item $D[X]$ is semiprime and semiprimitive.
		\item $D[\![X]\!]$ is semiprime and it is not semiprimitive.
	\end{enumerate}
\end{exercise}

% We will prove 
% in Lecture \ref{09} (Corollary \ref{cor:C[G]_semiprime}) 
% that the if $G$ is a group, 
% then the ring $\C[G]$ is semiprime. 

\begin{corollary}
\label{cor:C[G]_semiprime}
	The ring $\C[G]$ is semiprime.
\end{corollary}

\begin{proof}
	Since $J(\C[G])=\{0\}$ by Rickart's theorem and the Jacobson radical
	contains every nil ideal by Proposition~\ref{pro:nilJ}, it follows that
	$\C[G]$ does not contain non-trivial nil ideals. Thus $\C[G]$ does not
	contain non-trivial nilpotent ideals and hence $\C[G]$ is semiprime.
\end{proof}

\begin{exercise}
	Prove that $Z(\C[G])$ is semiprime.
\end{exercise}

We now characterize when complex group algebras 
are left artinian. For that purpose,
we need a lemma. This is similar to one of the implications proved in Proposition \ref{pro:semisimple}. However,
in the arbitrary setting we are considering, we need to use Zorn's lemma. 

\begin{lemma}
    Let $M$ be a semisimple module and $N$ be a submodule. 
    Then $N$ is a direct summand.
\end{lemma}

\begin{proof}[Sketch of the proof]
    Let $M=\oplus_{i\in I}M_i$ be a direct sum of simple submodules. 
    Since each $N\cap M_i$ is a submodule of $M_i$ and $M_i$ is simple, it follows
    that $N\cap M_i=\{0\}$ or $N\cap M_i=M_i$. If
    $N\cap M_i=M_i$ for all $i\in I$, then $N=M$ and the lemma is proved. So we may assume
    that there exists $i\in I$ such that $N\cap M_i=\{0\}$. Let $X$ be the set
    of subsets $J$ of $I$ such that $N\cap (\oplus_{j\in J}M_j)=\{0\}$. Our assumptions
    imply that $X$ is non-empty. Zorn's lemma implies the existence of 
    a maximal element $K$. Let $N_1=\oplus_{k\in K}M_k$. We claim that
    $N\oplus N_1=M$. If not, there exists $i\in I$ such that
    $M_i\not\subseteq N\oplus N_1$. The simplicity of $M_i$ implies that
    $M_i\cap (N\oplus N_1)=\{0\}$, which contradicts the maximality of $K$. 
\end{proof}

A direct application of the lemma proves that
complex group algebras of infinite groups are never semisimple. 

\begin{proposition}
    \label{pro:KGsemisimple}
    If $G$ is an infinite group, then $\C[G]$ is not semisimple. 
\end{proposition}

\begin{proof}
	Assume that $R=\C[G]$ is semisimple.  Let $I$ 
	be the augmentation ideal of $R$, that is
	\[
	I=\left\{\alpha=\sum_{g\in G}\lambda_gg\in R:\sum_{g\in G}\lambda_g=0\right\}.
	\]
	By the previous lemma, 
	there exists a non-zero left ideal $J$ such that 
	$R=I\oplus J$. Since $R$ is unitary, there exist $e\in I$ and $f\in J$ such that
	$1=e+f$. If
	$x\in I$, then $x=xe+xf$ and hence $xf=x-xe\in I\cap J=\{0\}$. Since 
	$x=xe$ for all $x\in I$, it follows that $e=e^2$. Similarly, one proves
	that $f^2=f$. Moreover, $ef=0$, as $ef\in I\cap J=\{0\}$.  Since $I$ 
	is the augmentation ideal of $R$ and $If=(Re)f=R(ef)=\{0\}$ 
 (note that $I=Re$ because $x=xe$ for all $x\in I$), we conclude that
	$(g-1)f=0$
	for all $g\in G$, as $g-1\in I$ for all $g\in G$. If $f=\sum_{h\in
	G}\lambda_hh$ (finite sum) and $g\in G$, then  
	\[
	f=gf=\sum_{h\in G}\lambda_h(gh)=\sum_{h\in
	G}\lambda_{g^{-1}h}h.
	\]
        Thus $\lambda_h=\lambda_{g^{-1}h}$ for all $g,h\in G$. Since $G$ 
	is infinite, some $\lambda_g=0$ and hence $f=0$. Thus $e=1$ and $I=\C[G]$, a contradiction. 
% 	If $f=0$, then $e=1$ and $I=\C[G]$, a contradiction.  
% 	, a contradiction because 
% 	$f\ne 0$ implies that the sum that defines $f$ should be an infinite sum.
\end{proof}

% The ideal $I(G)$ used in the proof of the previous proposition 
% is known as the \emph{augmentation ideal} 
% of $\C[G]$.

\begin{theorem}
	Let $G$ be a group. Then $\C[G]$ 
	is left artinian if and only if 
	$G$ is finite. 
\end{theorem}

\begin{proof}
    If $G$ is finite, then $\C[G]$ is left artinian because $\dim\C[G]=|G|<\infty$. So assume that 
    $G$ is infinite. By Rickart's theorem,   
	$J(\C[G])=0$. Moreover, $\C[G]$
	is not semisimple by the previous proposition. Thus
	$\C[G]$ is not left artinian by Theorem~\ref{thm:SSartin=J}.
\end{proof}




\subsection{Maschke's theorem}

We now present another instance of the semisimplicity problem.
In this case, our result is for finite groups. 

\begin{theorem}[Maschke]
\index{Maschke's theorem}
	Let $G$ be a finite group. Then $J(K[G])=\{0\}$ if and only 
	if the characteristic of $K$ is zero 
	or does not divide the order of $G$. 
\end{theorem}

\begin{proof}
	Assume that $G=\{g_1,\dots,g_n\}$, where $g_1=1$. Let 
	\[
	\rho\colon K[G]\to K,
	\quad
	\alpha\mapsto\trace(L_{\alpha}),
	\]
	where 
	$L_{\alpha}(\beta)=\alpha\beta$. Then 
	\[
	\rho(g_i)=\begin{cases}
	    n & \text{if $i=1$,}\\
	    0 & \text{if $2\leq i\leq n$},
	\end{cases}
	\]
	as $L_{g_i}(g_j)=g_{i}g_j\ne g_j$ if $i\ne j$ and hence the matrix of 
	$L_{g_i}$ in the basis $\{g_1,\dots,g_n\}$ contains zeros in the main diagonal.

	Assume that $J=J(K[G])$ is non-zero and let 
	$\alpha=\sum_{i=1}^n\lambda_ig_i\in J\setminus\{0\}$. Without loss of generality
	we may assume that $\lambda_1\ne 0$ (if $\lambda_1=0$ there exists some 
	$\lambda_i\ne 0$ and we need to take $g_i^{-1}\alpha\in J$). Then 
	\[
		\rho(\alpha)=\sum_{i=1}^n \lambda_i\rho(g_i)=n\lambda_1.
	\]
	Since $G$ is finite, $K[G]$ is a finite-dimensional algebra and hence 
	$K[G]$ is left artinian. Since $J$ is a nilpotent ideal, 
	in particular, $\alpha$ is a nilpotent element. Then 
	$L_{\alpha}$ is nilpotent and hence $0=\rho(\alpha)=n\lambda_1$. This implies that
	the characteristic of the field $K$ divides $n$. 

	Conversely, let $K$ be a field of prime characteristic and that this prime divides 
	$n$. Let $\alpha=\sum_{i=1}^ng_i$. Since $\alpha
	g_j=g_j\alpha=\alpha$ for all $j\in\{1,\dots,n\}$, the set 
	$I=K[G]\alpha$ is an ideal of $K[G]$. Since, moreover,   
	\[
		\alpha^2=\sum_{i=1}^n g_i\alpha=n\alpha=0
	\]
	in the field $K$, it follows that $I$ is a nilpotent non-zero ideal. Thus $J(K[G])\ne\{0\}$, 
	as Proposition~\ref{pro:nilJ} yields $I\subseteq J(K[G])$.
\end{proof}

Since the Jacobson radical of a group algebra of a finite group contains 
every nil left ideal, the following consequence of the theorem follows immediately:

\begin{corollary}
	\label{cor:GfinitoNOnil}
	Let $G$ be a finite group. Then $K[G]$ does not contain non-zero nil left ideals. 
\end{corollary}


% \begin{proof}
% 	Es consecuencia inmediata del teorema de Maschke ya que $J(K[G])$ contiene a
% 	todo ideal a izquierda nil.	
% \end{proof}

%\index{Anillo!semisimple}
%Recordemos que un anillo unitario $R$ se dice \emph{semisimple} si para cada
%ideal $I$ de $R$ existe un ideal $J$ de $R$ tal que $R=I\oplus J$.
%
%%\begin{corollary}
%%	Sea $G$ un grupo finito y $K$ un cuerpo de característica coprima con el
%%	orden de $G$. Entonces $K[G]$ es semisimple.
%%\end{corollary}
%%
%%\begin{proof}
%%	
%%\end{proof}
%
%\begin{theorem}
%	Si $G$ es un grupo infinito, entonces $K[G]$ nunca es semisimple.
%\end{theorem}
%
%\begin{proof}
%	Sea $R=K[G]$ y supongamos que $R$ es semisimple.  Si $I$ es el ideal de
%	aumentación de $R$, existe un ideal no nulo $J$ de $R$ tal que $R=I\oplus
%	J$. Como $R$ es unitario, existen $e\in I$, $f\in J$ tales que $1=e+f$. Si
%	$x\in I$, entonces $x=xe+xf$ y luego $xf=x-xe\in I\cap J=\{0\}$. Como
%	entonces $x=xe$ para todo $x\in I$, en particular $e_1=e_1^2$. Análogamente
%	vemos que $e_2^2=e_2$. Además $ef=0$ pues $ef\in I\cap J=\{0\}$.  Como $I$
%	es el ideal de aumentación y $If=(Re)f=R(ef)=0$, se concluye que $(g-1)f=0$
%	para todo $g\in G$ pues $g-1\in I$. Si suponemos que $f=\sum_{h\in
%	G}\lambda_hh$, entonces 
%	\[
%	f=gf=\sum_{h\in G}\lambda_h(gh)=\sum_{h\in
%	G}\lambda_{g^{-1}h}h.
%	\]
%	Luego $\lambda_h=\lambda_{g^{-1}h}$ para todo $g,h\in G$, una contradicción
%	pues como $f\ne 0$ la suma que define a $f$ es infinita. 
%\end{proof}


\subsection{Herstein's theorem}

Our aim now is to answer the following question: When
a group algebra is algebraic? Herstein's theorem provides
a solution in the case of fields of characteristic zero. In prime characteristic,
the problem is still open. 

\begin{definition}
\index{Group!locally finite}
	A group $G$ is \emph{locally finite} if every finitely generated 
	subgroup of $G$ is finite. 
\end{definition}

If $G$ is a locally finite group, then every element $g\in G$ has finite order, as
the subgroup $\langle g\rangle$ is finite because it is finitely generated.

\begin{example}
    Every finite group is locally finite
\end{example}

\begin{example}
    The group $\Z$ is not locally finite because it is torsion-free.
\end{example}

\begin{example}
\index{Pr\"ufer's group}
	Let $p$ be a prime number. 
	The \emph{Pr\"ufer's group}  
	\[
		\Z(p^{\infty})=\{z\in\C:z^{p^n}=1\text{ for some $n\in\Z_{>0}$}\}, 
	\]
	is locally finite. 
\end{example}

\begin{example}
	Let $X$ be an infinite set and $\Sym_X$ be the set of bijective maps $X\to
	X$ moving only finitely many elements of $X$. Then 
	$\Sym_X$ is locally finite.
\end{example}

\index{Group!torsion}
A group $G$ is a \emph{torsion} group if every element of $G$
has finite order. Locally finite groups are torsion groups. 

\begin{example}
    Abelian torsion groups are locally finite. Let $G$ be a locally finite abelian group 
    and $H$ be a finitely generated subgroup. Since $G$ is an abelian torsion group, so is $H$. Thus
    $H$ is finite by the structure theorem of abelian groups. 
\end{example}

\begin{proposition}
\label{pro:exact_LI}
	Let $G$ be a group and $N$ be a normal subgroup of $G$. If $N$ and $G/N$
	are locally finite, then $G$ is locally finite.
\end{proposition}

\begin{proof}
	Let $\pi\colon G\to G/N$ be the canonical map and $\{g_1,\dots,g_n\}$ be a finite subset of $G$. 
	Since $G/N$ is locally finite, the subgroup $Q$ of $G/N$ generated by 
	$\pi(g_1),\dots,\pi(g_n)$ is finite, say
	\[
		Q=\{\pi(g_1),\dots,\pi(g_n),\pi(g_{n+1}),\dots,\pi(g_m)\}
	\]
	for some $g_{n+1},\dots,g_m\in G$. 
	
	For each $i,j\in\{1,\dots,n\}$ there exist $u_{ij}\in N$ and 
	$k\in\{1,\dots,m\}$ such that \[
	g_ig_j=u_{ij}g_k.
	\]
	Let $U$ be the subgroup of $N$
	generated by $\{u_{ij}:1\leq i,j\leq n\}$. Since $N$ is locally finite, $U$ is finite. Moreover, since 
	each $g_ig_jg_l$ can be written as 
	\[
		g_ig_jg_l=u_{ij}g_kg_l=u_{ij}u_{kl}g_t=ug_t
	\]
	for some $u\in U$ and $t\in\{1,\dots,m\}$, it follows that the subgroup 
	$H$ of $G$ generated by $\{g_1,\dots,g_n\}$ is finite, as 
	$|H|\leq m|U|$. 
\end{proof}

\index{Group!solvable}
A group $G$ is
\emph{solvable} if there exists a sequence
of subgroups 
\begin{equation}
	\label{eq:resoluble}
	\{1\}=G_0\subsetneq G_1\subsetneq \cdots\subsetneq G_n=G
\end{equation}
where each $G_i$ is normal in $G_{i+1}$ and each 
quotient $G_i/G_{i-1}$ is
abelian.

\begin{example}
    Abelian groups are solvable. 
\end{example}

Subgroups and quotients of solvable groups are solvable. 

\begin{example}
    Groups of order $<60$ are solvable.
\end{example}

\begin{example}
    $\Alt_5$ and $\Sym_5$ are not solvable. 
\end{example}

\index{Burnside's theorem}
\index{Feit--Thompson's theorem}
A famous theorem of Burnside states that 
groups of order $p^aq^b$ for prime numbers $p$ and $q$ are solvable 
A much harder theorem proved by Feit and Thompson states that
groups of odd order are solvable.

\begin{proposition}
	If $G$ is a solvable torsion group, 
	then $G$ is locally finite. 
\end{proposition}

\begin{proof}
	We proceed by induction on $n$, the length of the sequence~\eqref{eq:resoluble}. 
	If $n=1$, then $G$ is finite because it is abelian and a torsion group.
	Now assume the result holds for solvable groups of length $n-1$ and let
	$G$ be a solvable group with a sequence~\eqref{eq:resoluble}. Since $G_{n-1}$ is 
	a solvable torsion group, the inductive hypothesis implies that 
	$G_{n-1}$ is locally finite. Since $G/G_{n-1}$ is an abelian torsion group, 
	it is locally finite. The result now follows from Proposition \ref{pro:exact_LI}.
\end{proof}

We now prove Herstein's theorem.

\begin{theorem}[Herstein]
\index{Herstein's theorem}
	If $G$ is a locally finite group, then $K[G]$ is algebraic. Conversely, if 
	$K[G]$ is algebraic and $K$ has characteristic zero, then $G$ 
	is locally finite. 
\end{theorem}

\begin{proof}
	Assume that $G$ is locally finite. Let $\alpha\in K[G]$. The subgroup 
	$H=\langle\supp\alpha\rangle$ is finite, as it is finitely generated. Since 
	$\alpha\in K[H]$ and $\dim_KK[H]<\infty$, the set 
	$\{1,\alpha,\alpha^2,\dots\}$ is linearly dependent. Thus $\alpha$ is
	algebraic over $K$.

	Let $\{x_1,\dots,x_m\}$ be a finite subset of $G$. Adding inverses if needed,
	we may assume that $\{x_1,\dots,x_m\}$ generates the subgroup 
	$H=\langle x_1,\dots,x_m\rangle$ as a semigroup. Let 
	\[
	\alpha=x_1+\dots+x_m\in K[G].
	\]
	Since $\alpha$ is algebraic over $K$, 
	there exist $b_0,b_1,\dots,b_{n+1}\in K$ such that 
	\[
	b_0+b_1\alpha+\cdots+b_{n+1}\alpha^{n+1}=0,
	\]
	where $b_{n+1}\ne 0$. We can rewrite this as 
	\[
		\alpha^{n+1}=a_0+a_1\alpha+\cdots+a_n\alpha^n
	\]
	for some $a_0,\dots,a_n\in K$. 
	Note that 
	\[
		\alpha^{k}=(x_1+\cdots+x_m)^{k}
		=\sum x_{i_1}\cdots x_{i_{k}}
	\]
	for all $k$. 
	Two words $x_{i_1}\cdots x_{i_{k}}$ and 
	$x_{j_1}\cdots x_{j_{l}}$ could represent the same element of the group $H$. In this case, 
	the coefficient of $x_{i_1}\cdots x_{i_{k}}=x_{j_1}\cdots x_{j_{l}}$ 
	in $\alpha^{k}$ will be a positive integer $\geq2$. 
 
        Let $w=x_{i_1}\cdots
	x_{i_{n+1}}\in H$ be a word of length $n+1$. 
		Since $K$
	is of characteristic zero, it follows that $w\in\supp(\alpha^{n+1})$. Since, moreover,  
	$\alpha^{n+1}=\sum_{j=0}^na_j\alpha^j$, it follows that 
	$w\in\supp(\alpha^j)$ for some $j\in\{0,\dots,n\}$. Thus each
	word in the letters $x_j$ of length $n+1$ can be written as a word in the letters $x_j$ of 
	length $\leq n$. Therefore $H$ is finite and hence $G$ is locally finite. 
\end{proof}

