\section{Lecture: Week 1}

\subsection{Semisimple algebras and the Artin--Wedderburn theorem}

We will study  
finite-dimensional semisimple algebras. The main goal of the first two lectures is to
prove the Artin--Wedderburn theorem. 

\begin{definition}
	\index{Algebra}
	An \emph{algebra} (over the field $K$) is a vector space (over $K$) 
	with an associative multiplication $A\times A\to A$ such that
	\[
    a(\lambda b+\mu c)=\lambda(ab)+\mu(ac)\quad\text{and}\quad 
	(\lambda a+\mu b)c=\lambda(ac)+\mu (bc)
 \]
    for all $a,b,c\in A$, and 
	that contains an element $1_A\in A$ such that 
 \[
 1_Aa=a1_A=a
 \]
 for all $a\in A$.
 \end{definition}

Note that a ring $A$ is an algebra over $K$ 
if and only if 
there is a ring homomorphism $K\to Z(A)$, where $Z(A)=\{a\in A:ab=ba\text{ for all $b\in A$}\}$ is the \emph{center} of $A$, 
such that $1_K\to 1_A$. 
% n algebra $A$ over $K$ is, in particular, a ring $A$ 
% such that is a vector space
% (over $K$) such that the map $K\to A$, $\lambda\mapsto \lambda1_A$, is injective. 

\begin{definition}
	\index{Algebra!commutative}
	An algebra $A$ is \emph{commutative} if $ab=ba$ for all $a,b\in A$. 
\end{definition}

\index{Algebra!dimension}
The \emph{dimension} of an algebra $A$ is the dimension of $A$ as a vector space. This is why we want to consider algebras, as 
they are a linear version of rings. Often, our arguments will use the dimension of the underlying vector space.  

\begin{example}
	The field $\R$ is a real algebra and  
	$\C$ is a complex algebra. Moreover, $\C$ is a real algebra. 
\end{example}

Any field $K$ is an algebra over $K$.

\begin{example}
	If $K$ is a field, then $K[X]$ is an algebra over $K$. 
\end{example}

Similarly, the polynomial ring $K[X,Y]$ and the ring $K[\![X]\!]$ of power series
are examples of algebras over $K$. 

\begin{example}
	If $A$ is an algebra, then  $M_n(A)$ is an algebra. 
\end{example}

\begin{example}
	The set of continuous maps $[0,1]\to\R$ is a real algebra with the usual
	point-wise operations $(f+g)(x)=f(x)+g(x)$ and $(fg)(x)=f(x)g(x)$. 
\end{example}

\begin{example}
	Let $n\in\Z_{>0}$. Then $K[X]/(X^n)$ is a finite-dimensional $K$-algebra. 
    It is the \emph{truncated polynomial algebra}.  
\end{example}

\begin{example}
	Let $G$ be a finite group. The vector space 
	$\C[G]$ with basis $\{g:g\in G\}$
	is an algebra with multiplication
	\[
	\left(\sum_{g\in G}\lambda_gg\right)\left(\sum_{h\in G}\mu_hh\right)
	=\sum_{g,h\in G}\lambda_g\mu_h(gh).
	\] 	
	Note that $\dim\C[G]=|G|$ and
	$\C[G]$ is commutative if and only $G$ is abelian. 
	This is the \emph{complex group algebra} of $G$. 
\end{example}

If $G$ is an infinite group, the complex group algebra $\C[G]$ 
is defined as the set 
of finite linear combinations of elements of $G$ 
with the usual operations. 

\begin{definition}
    \index{Homomorphism!of algebras}
    Let $K$ be a field and $A$ and $B$ be $K$-algebras. 
    An algebra \emph{homomorphism} is a ring homomorphism $f\colon A\to B$ that is also a $K$-linear map. 
\end{definition}

The complex conjugation map  
$\C\to \C$, $z\mapsto\overline{z}$, is a ring homomorphism that is not an algebra homomorphism over $\C$. 
 
\begin{exercise}
\label{xca:G_zero_divisors}
	Let $G$ be a non-trivial finite group. 
	Then $\C[G]$ has zero divisors. 
\end{exercise}

If $A$ is an algebra, then $\mathcal{U}(A)$ is the set 
of units of $A$. 

\begin{exercise}
\label{xca:units_UP}
	Let $A$ be a $K$-algebra and $G$ be a finite group. 
	If $f\colon G\to\mathcal{U}(A)$ is a group homomorphism, 
	then there exists an algebra homomorphism 
	$\varphi\colon K[G]\to A$ such that $\varphi|_G=f$.   	
\end{exercise}

\begin{definition}
 	\index{Algebra!ideal}
 	An \emph{ideal} of an algebra is an ideal of the underlying ring.
\end{definition}

Similarly, one defines left and right ideals of an algebra.

If $A$ is an algebra, then every left ideal of the ring $A$ is a vector space.  
Indeed, if $I$ is a left ideal of $A$ 
and $\lambda\in K$ and $x\in I$, then 
\[
	\lambda x=\lambda (1_Ax)=(\lambda 1_A)x.
\]
Since $\lambda 1_A\in A$, it follows that  $\lambda I=(\lambda
1_A)I\subseteq I$. 
Similarly, every right ideal of the ring $A$ is a vector space. 

If $A$ is an algebra and $I$ is an ideal of $A$, then the quotient ring $A/I$ has a unique algebra
structure such that the canonical map  
$A\to A/I$, $a\mapsto a+I$, is a surjective algebra homomorphism with kernel~$I$. 

\begin{definition}
    \index{Algebraic element}
    \index{Algebra!algebraic}
    Let $A$ be an algebra over the field $K$. An element $a\in A$ is 
    \emph{algebraic} over $K$ if there exists a non-zero polynomial $f\in K[X]$
    such that $f(a)=0$. 
\end{definition}

If every element of $A$ is algebraic, then $A$ is said to be \emph{algebraic} 

In the algebra $\R$ over $\Q$, the element $\sqrt{2}$ is algebraic, as $\sqrt{2}$ is a root of the polynomial $X^2-2\in\Q[X]$. A famous theorem of Lindemann proves that $\pi$ is not algebraic over $\Q$. Every element of the real algebra $\R$ is algebraic.

\begin{proposition}
	\label{lem:algebraic}
	Every finite-dimensional algebra is algebraic.
\end{proposition}

\begin{proof}
   Let $A$ be an algebra with $\dim A=n$ and let $a\in A$. Since  
	$\{1,a,a^2,\dots,a^n\}$ is linearly dependent (because if it does not have repetitions, has $n+1$ elements), there exists 
	a non-zero polynomial $f\in K[X]$ such that $f(a)=0$.
\end{proof}

\begin{definition}
\index{Module}
    A \emph{module} over an algebra $A$ is a module 
    over the ring $A$.
\end{definition}

Similarly, one defines \emph{submodules} of $A$-modules. 

\begin{definition}
\index{Module!homomorphism}
Let $A$ be a $K$-algebra. 
A \emph{homomorphism} of $A$-modules 
$f\colon M\to N$ is a $K$-linear map such that $f(a\cdot m)=a\cdot f(m)$ for all
$a\in A$ and $m\in M$.  
\end{definition}

It is a straightforward exercise to 
prove the isomorphism theorems. 

Let $A$ be a finite-dimensional $K$-algebra. 
If $M$ is a module over the ring $A$, then $M$ is a vector space with  
\[
\lambda m=(\lambda 1_A)\cdot m, 
\]
where $\lambda\in K$ and $m\in M$. 
Moreover, $M$ is finitely generated if and only if $M$ is finite-dimensional.  

\begin{example}
If $M$ is a module over a finite-dimensional $K$-algebra $A$, one defines $\End_A(M)$ as the set
of $A$-module homomorphisms $M\to M$. The set  
$\End_A(M)$ is indeed a $K$-algebra with 
\[
(f+g)(m)=f(m)+g(m),\quad 
(\lambda f)(m)=\lambda f(m)
\quad\text{and}
\quad 
(fg)(m)=f(g(m))
\]
for all $f,g\in\End_A(M)$, 
$\lambda\in K$ and $m\in M$. 
\end{example}

% In this chapter we will work with finitely generated modules. 

\begin{example}
An algebra  $A$ is a module over $A$ with left multiplication, that is 
\[
a\cdot b=ab,\quad a,b\in A.
\]
This module is the (left) \emph{regular representation} of $A$ and it will be denoted by $\prescript{}{A}{A}$. 
\end{example}

\begin{definition}
\index{Module!simple}
	Let $A$ be an algebra and $M$ be a module over $A$. Then 
	$M$ is \emph{simple} if $M\ne\{0\}$ and $\{0\}$ and $M$ 
	are the only submodules of $M$.	
\end{definition}

\begin{definition}
\index{Module!semisimple}
	Let $A$ be a finite-dimensional 
	algebra and $M$ be a finite-dimensional module over $A$. Then 
	$M$ is \emph{semisimple} if $M$ is a direct sum of 
	finitely many simple submodules.  
\end{definition}

By definition, the zero module is semisimple. Moreover, 
any finite direct sum of semisimples is semisimple. 

% The zero module is semisimple.  

\begin{lemma}[Schur]
\index{Schur's lemma}
	Let $A$ be an algebra. If $S$ and $T$ are
	simple modules and $f\colon S\to T$ is a non-zero $A$-module homomorphism, 
	then $f$ is an isomorphism. 
\end{lemma}

\begin{proof}
Since $f\ne 0$, $\ker f$ is a proper submodule of $S$. Since $S$ is simple, it follows 
that $\ker f=\{0\}$. Similarly, $f(S)$ 
is a non-zero submodule of $T$ and hence $f(S)=T$, as $T$ is simple. 	
\end{proof}

\begin{proposition}
    If $A$ is a finite-dimensional algebra and $S$ is a simple $A$-module, then $S$ is finite-dimensional. 
\end{proposition}

\begin{proof}
    Let $s\in S\setminus\{0\}$. Since $S$ is simple, $\varphi\colon A\to S$, $a\mapsto a\cdot s$, is a surjective 
    module homomorphism. 
    In particular, by the first isomorphism theorem, 
    $A/\ker\varphi\simeq S$ and hence 
    \[
    \dim S=\dim (A/\ker\varphi)\leq \dim A.\qedhere
    \]
\end{proof}

\begin{proposition}
\label{pro:semisimple}
	Let $A$ be a finite-dimensional algebra and $M$ a finite-dimensional $A$-module. The following statements are equivalent:
	\begin{enumerate}
		\item $M$ is semisimple.
		\item $M=\sum_{i=1}^k S_i$, where each $S_i$ is a simple submodule of $M$. 
		\item If $S$ is a submodule of $M$, then there is a submodule $T$ of $M$ such that $M=S\oplus T$.    
	\end{enumerate}
\end{proposition}

\begin{proof}
	We first prove that $2)\implies3)$.
	Let $N\ne\{0\}$ be a submodule of $M$. Since $N\ne\{0\}$ and $\dim M<\infty$, there exists a submodule 
	$T$ of $M$ of maximal dimension such that 
	$N\cap T=\{0\}$. If $S_i\subseteq N\oplus T$ for all $i\in\{1,\dots,k\}$, then, as $M$ is the sum of the $S_i$, it follows that
	$M=N\oplus T$. 
	If, however, there exists $i\in\{1,\dots,k\}$ such that $S_i\not\subseteq N\oplus T$, then $S_i\cap (N\oplus T)\subseteq S_i$. 
	Since the module $S_i$ is simple,
	it follows that $S_i\cap (N\oplus T)=\{0\}$. Thus $N\cap (S_i\oplus T)=\{0\}$, a contradiction to the maximality of
	$\dim T$.  
	
	The implication  $1)\implies2)$ is trivial. 
	
	Finally, we prove that $3)\implies1)$. 
	We proceed by induction on $\dim M$. The result is clear if $\dim M=1$. Assume that $\dim M\geq2$ and   
	let $S$ be a non-zero submodule of $M$ of minimal dimension. In particular, $S$ is simple. 
    By assumption, there exists a submodule $T$ of $M$ such that $M=S\oplus T$. We claim that $T$ satisfies the assumptions. 
	If $X$ is a submodule of $T$, then, since $T$ is also a submodule of $M$, there exists a submodule $Y$ of $M$ such that 
	$M=X\oplus Y$. Thus  
	\[
	T=T\cap M=T\cap (X\oplus Y)=X\oplus (T\cap Y),
	\]
	as $X\subseteq T$. 
	Since $\dim T<\dim M$ and $T\cap Y$ is a submodule of $T$, the inductive hypothesis implies 
	that $T$ is a direct sum of simple submodules. Hence $M$ is a direct sum of simple submodules. 
\end{proof}

\begin{proposition}
    If $M$ is a semisimple module and $N$ is a submodule, then $N$ and $M/N$ are semisimple.	
\end{proposition}

\begin{proof}
	Assume that $M=S_1+\cdots+ S_k$, where each $S_i$ is a simple submodule. If $\pi\colon M\to M/N$ 
	is the canonical map, the techniques used in Schur's lemma imply that each restriction $\pi|_{S_i}$ 
	is either zero or an isomorphism with the image. Since  
	\[
	M/N=\pi(M)=\sum_{i=1}^k(\pi|_{S_i})(S_i),
	\]
	it follows that $M/N$ is a direct sum of finitely many simples. 
	
	We now prove that $N$ is semisimple. By assumption, 
	there exists a submodule $T$ of $M$ such that 
	$M=N\oplus T$. The quotient
	$M/T$ is semisimple by the previous paragraph, so it follows that 
	\[
	N\simeq N/\{0\}=N/(N\cap T)\simeq (N\oplus T)/T=M/T
	\]
	is also semisimple.     
\end{proof}

\begin{definition}
\index{Algebra!semisimple}
    An algebra $A$ is \emph{semisimple} if every finitely generated $A$-module is semisimple. 
\end{definition}

\begin{proposition}
Let $A$ be a finite-dimensional algebra. Then $A$ is semisimple if and only if 
the regular representation of $A$ is semisimple. 
\end{proposition}

\begin{proof}
Let us prove the non-trivial implication. Let $M$ be a finitely generated module, say $M=(m_1,\dots,m_k)$. 
The map
\[
\bigoplus_{i=1}^k A\to M,\quad
(a_1,\dots,a_k)\mapsto \sum_{i=1}^k a_i\cdot m_i,
\]
is a surjective homomorphism of modules, where $A$ is considered as a module with the regular
representation. Since 
$A$ is semisimple, it follows that $\bigoplus_{i=1}^kA$ is semisimple. 
Thus $M$ is semisimple, as it is isomorphic to the quotient of a semisimple module.
\end{proof}

\begin{theorem}
Let $A$ be a finite-dimensional semisimple algebra. 
Assume that the regular representation can be decomposed as $A=\bigoplus_{i=1}^k S_i$ where each $S_i$ is a simple submodule.  
If $S$ is a simple module, then $S\simeq S_i$ for some $i\in\{1,\dots,k\}$. 
\end{theorem}

\begin{proof}
Let $s\in S\setminus\{0\}$. The map $\varphi\colon A\to S$, $a\mapsto a\cdot s$, is a surjective module homomorphism. Since 
$\varphi\ne 0$, there exists $i\in\{1,\dots,k\}$ such that some restriction 
$\varphi|_{S_i}\colon S_i\to S$ is non-zero. By Schur's lemma, it follows that  
$\varphi|_{S_i}$ is an isomorphism.   	
\end{proof}

As a corollary, a finite-dimensional semisimple algebra admits 
only finitely many isomorphism classes of simple modules. When we say that 
the $S_1,\dots,S_k$ are the simple modules of an algebra, this means that the $S_i$ are the representatives
of isomorphism classes of all simple modules of the algebra, that is that each simple module is isomorphic to
some $S_i$ and, moreover,  
$S_i\not\simeq S_j$ whenever $i\ne j$. 
