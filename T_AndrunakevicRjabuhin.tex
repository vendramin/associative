\section{Project: The Andrunakevic--Rjabuhin theorem}
\label{section:AndrunakevicRjabuhin}
%\begin{exercise}
%\label{xca:reduced}
%     Let $R$ be a ring and $I$ be an ideal of $R$.
%     Prove that $I$ is prime if and only if $xRy\subseteq I$ implies
%     either $x\in I$ or $y\in I$. 
% \end{exercise}

% \begin{sol}{xca:reduced}
%     Let $A$ and $B$ be ideals such that $AB\subseteq I$. If 
%     $A\not\subseteq I$ and $B\not\subseteq J$, let 
%     $x\in A\setminus P$ and $y\in B\setminus P$. Then
%     $xRy\subseteq AB\subseteq I$, a contradiction. 
%     Conversely, if $xRy\subseteq I$ and $x\not I$ and $y\not\in I$, 
%     then $A=(x)\not\subseteq I$ and $B=(y)\not\subseteq P$.
% \end{sol}

\begin{definition}
\index{Ring!reduced}
     A ring $R$ is \emph{reduced} if 
     has no non-zero nilpotent elements. 
\end{definition}

Every commutative domain is reduced. 

\begin{example}
    The ring $\Z\times\Z$ with the usual operations 
    is reduced but not a domain. 
\end{example}

\begin{example}
    The ring $\Z/6$ is reduced. However, $\Z/4$ is not reduced. 
\end{example}

\begin{exercise}
\label{xca:reduced}
    Prove that a ring $R$ is \emph{reduced} if and only 
    if for all $r\in R$ such that $r^2=0$ one has $r=0$.
\end{exercise}

%\begin{exercise}
%\label{xca:reduced_Zn}
%    Let $n\geq2$. Then $\Z/n$ is reduced 
%    but not a domain if and only if $n$ is square-free 
%    but not prime.
%\end{exercise}

\begin{exercise}
    \label{xca:reduced_RX}
    Let $R$ be a commutative ring that is reduced but not a domain.
    Prove that $R[X]$ is reduced but not a domain. 
\end{exercise}

The previous exercise and induction 
shows that if $R$ is reduced but not a domain, 
then so is $R[X_1,\dots,X_n]$. 

\begin{example}
    Let $R=\Z/3\times\Z/3$ with
    operations 
    \[
    (a,b)+(c,d)=(a+c,b+d),\quad 
    (a,b)(c,d)=(ac,ad+bc).
    \]
    Then $R$ is a commutative
    ring with identity $(1,0)$. Since 
    $(0,1)$ is a non-zero nilpotent element, $R$ is not reduced. 
\end{example}

\begin{definition}
\index{Ideal!reduced}
    Let $R$ be a ring and $I$ be an ideal of $R$. 
    Then $I$ is \emph{reduced} if $R/I$ is a reduced ring. 
\end{definition}

Let $R$ be a ring and 
$I$ be a reduced ideal of $R$. If $ab\in I$, then 
$ba\in I$. In fact, since $ab\in I$, 
$(ba)^2=b(ab)a\in I$.
Since $R/I$ is reduced, $ba\in I$. 

 \begin{theorem}[Andrunakevic--Rjabuhin]
 \label{thm:AndrunakevicRjabuhin}
 \index{Andrunakevic--Rjabuhin theorem}
 	Let $R$ be a non-zero ring. If $R$ is reduced, there exists
 	an ideal $I$ of $R$ such that 
 	then $R/I$ has no non-zero zero-divisors. 
 \end{theorem}

 Let $R$ be a ring and $I$ be an ideal of $R$. If $S$ 
 is a subset of $R$, the \emph{left annihilator} of $S$
 modulo $I$ is the set $\{r\in R:rS\subseteq I\}$.  

 \begin{lemma}
    Let $R$ be a ring and $I$ be a reduced ideal. 
    If $S\subseteq R$ is a subset, then 
    the left annihilator of $S$ modulo $I$ 
    is a reduced ideal. 
 \end{lemma}

\begin{proof}
    We need to show that $A=\{r\in R:rS\subseteq I\}$ 
    is a reduced ideal. 
    A straightforward calculation shows that $A$ is 
    a left ideal. We claim that $A$ is a right ideal. Let $r\in R$
    and $a\in A$. Then 
    $as\in I$ for all $s\in S$. Since $I$ is reduced, $sa\in I$ for all $s\in S$. Since 
    $I$ is an ideal of $R$, $sar\in I$
    for all $s\in S$. Using again 
    that $I$ is reduced, 
    $ars\in I$ for all $s\in S$. Thus 
    $ar\in A$. 
    
    We now claim that $A$ is reduced. If $a^2\in A$, then 
    $aas=a^2s\in I$ for all $s\in S$. 
    Since $I$ is reduced, $asa\in I$ for
    all $s\in S$. Thus $(as)^2=(asa)s\in I$ for all $s\in S$. 
    Since $I$ is reduced, $as\in I$ for all $s\in S$. Hence $a\in A$. 
\end{proof}

Similarly, if $S$ is a subset of a ring $R$, then 
the \emph{right annihilator} 
$\{r\in R:Sr\subseteq I\}$ 
of $S$ modulo $I$ 
is a reduced ideal. 

\begin{proof}[Proof of Theorem \ref{thm:AndrunakevicRjabuhin}]
    Let $x\in R\setminus\{0\}$. Let $X$ 
    be the set of reduced ideals $I$ such that 
    $x\not\in I$. Since $R$ is reduced, $\{0\}$ 
    is a reduced ideal and hence $X\ne\emptyset$. 
    A standard application of Zorn's lemma shows that
    there exists a maximal element $M\in X$. 
    
    We claim that $R/M$ has no non-zero divisors. If not, 
    there exist $a,b\in R$ such that $ab\in M$, $a\not\in M$ 
    and $b\not\in M$. Let $A$ be the left annihilator of $\{b\}$ 
    modulo $M$ and $B$ be the right annihilator of $\{a\}$ 
    modulo $M$. By the previous lemma, $A$ and $B$ 
    are reduced ideals of $R$. Since  
    $a\in A$, $M\subsetneq A$. Similarly, since 
    $b\in B$, $M\subsetneq B$. Moreover, $AB\subseteq M$. 
    Since $x\in A\cap B$, $x^2\in AB\subseteq M$. Since 
    $M$ is reduced, $x\in M$, a contradiction. 
\end{proof}

\begin{exercise}
    Prove that a reduced ring is a subdirect product
    of rings without no non-zero divisors. 
\end{exercise}
