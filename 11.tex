\chapter{}
\label{11}

\topic{Prime rings}

In commutative algebra, domains play a fundamental role. In non-commutative
algebra certain things could be quite different. 
For example, the ring $M_n(\C)$ is not a domain.
We need a non-commutative generalization of domains.

\begin{definition}
	\index{Ring!prime} 
	Let $R$ be a ring (not necessarily with one). Then $R$ is
	\textbf{prime} if for $x,y\in R$ such that $xRy=\{0\}$ it follows that $x=0$ or 
	$y=0$.
\end{definition}

\begin{example}
    \index{Domain}
	A ring $R$ is a \textbf{domain} if $xy=0$ implies
	$x=0$ or $y=0$. Each domain is trivially a prime ring.
\end{example}

\begin{example}
    A commutative ring is prime if and only if it is a domain, as $ab=0$ 
    if and only if $aRb=\{0\}$.
\end{example}

\begin{example}
    A non-zero ideal of a prime ring is a prime ring.
\end{example}

\begin{exercise}
\label{xca:domain<=>prime+reduced}
    A ring is a domain if and only if
    it is both prime and reduced. 
\end{exercise}

A characterization of prime rings:

\begin{proposition}
    Let $R$ be a ring. The following statements are equivalent:
	\begin{enumerate}
		\item $R$ is prime.
		\item If $I$ and $J$ are left ideals such that $IJ=\{0\}$, then 
			$I=\{0\}$ or $J=\{0\}$.
		\item If $I$ and $J$ are ideals such that $IJ=\{0\}$, then $I=\{0\}$ or
			$J=\{0\}$.
	\end{enumerate}
\end{proposition}

\begin{proof}
	We first prove that $1)\implies2)$. Let $I$ and $J$ be left ideals such that
	$IJ=\{0\}$. Then $IRJ=I(RJ)\subseteq IJ=\{0\}$. If $J\ne
	\{0\}$, $u\in I$ and $v\in J\setminus\{0\}$, then $uRv\in IRJ=\{0\}$. Hence 
	$u=0$.

	The implication $2)\implies3)$ is trivial. 

    Let us prove that $3)\implies1)$. Let $x,y\in R$ be such that $xRy=\{0\}$.
	Let $I=RxR$ and $J=RyR$. Since $IJ=(RxR)(RyR)=R(xRy)R=\{0\}$, 
	we may assume that $I=\{0\}$. In particular, $Rx$ and $xR$ are ideals, as 
	$R(xR)=(Rx)R=\{0\}$. Then $\Z x$ is an ideal of $R$ such that $(\Z x)R=\{0\}$. 
	Thus $x=0$. 
\end{proof}

\begin{theorem}[Connel]
\index{Connel's theorem}
\label{thm:Connel}
    Let $K$ be a field of characteristic zero and
    $G$ be a group. Then $K[G]$ is prime
    if and only if $G$ does not contain 
    non-trivial finite normal subgroups. 
\end{theorem}

\begin{proof}
    See for example \cite[Theorem 2.10 of Chapter 4]{MR798076}.
\end{proof}

\begin{corollary}[Connel]
\label{cor:Connel}
    Let $K$ be a field of characteristic zero and 
    $G$ be a group. Then $K[G]$ is left artinian if and only
    if $G$ is finite. 
\end{corollary}

\begin{proof}
    It follows from Theorem \ref{thm:Connel} and
    Hopkins--Levitzky theorem; see 
    \cite[Theorem 1.1 of Chapter 10]{MR798076}. 
\end{proof}

Simple rings are trivially prime. The converse is not true:

\begin{example} 
    $\Z$ is a domain, so it is a prime ring. Clearly, it is not simple.
\end{example}

\begin{example}
	If $R_1$ and $R_2$ are rings, $R=R_1\times R_2$ is not prime, as 
	$I=R_1\times\{0\}$ and $J=\{0\}\times R_2$ are non-zero ideals such that $IJ=\{0\}$.
\end{example}

\begin{lemma}
	\label{lem:primoizqmin=>prim}
	Let $R$ be a prime ring and $L$ be a minimal left ideal of $R$.
	Then $R$ is primitive. 
\end{lemma}

\begin{proof}
	Since $L$ is a minimal left ideal, it is simple as a module over $R$. 
	We claim that $L$ is faithful. Let $y\in L\setminus\{0\}$ and
	$x\in\Ann_R(L)$. Since $xRy\in xRL\subseteq xL=\{0\}$, it follows that 
	$x=0$.
\end{proof}

\begin{lemma}
	\label{lem:denso_artiniano}
	Let $D$ be a division ring and $R$ be a dense ring in a module $V$ over $D$. 
	If $R$ is left artininian, then $\dim_DV<\infty$.
\end{lemma}

\begin{proof}
	Assume that $\dim_DV=\infty$ and let $\{u_1,u_2,\dots,\}$ be linearly independent. 
	Since $R\subseteq\End_D(V)$, it follows that 
	$V$ is a module over $R$ with $f\cdot v=f(v)$, where $f\in R$ y $v\in V$. 
	For $n\in\Z_{>0}$ let 
	\[
		I_n=\Ann_R(\{u_1,\dots,u_n\}).
	\]
	Each $I_j$ is a left ideal of $R$ and $I_1\supseteq
	I_2\supseteq\cdots\supseteq I_n\supseteq\cdots$. Let 
	$n\in\Z_{>0}$ and $v\in V\setminus\{0\}$. Since $R$ is dense
	in $V$, there exists $f\in R$ such that $f(u_j)=0$ for all $j\in\{1,\dots,n\}$ and 
	$f(u_{n+1})=v\ne0$. Thus $I_1\supsetneq I_2\supsetneq\cdots\supsetneq
	I_n\supsetneq\cdots$, a contradiction.
\end{proof}

\begin{theorem}[Wedderburn]
	\index{Wedderburn's theorem}
	Let $R$ be a left artinian ring. The following statements are equivalent:
	\begin{enumerate}
		\item $R$ is simple.
		\item $R$ is prime.
		\item $R$ is primitive.
		\item $R\simeq M_n(D)$ for some $n$ and some division ring $D$.
	\end{enumerate}
\end{theorem}

\begin{proof}
	The implication $1)\implies2)$ is trivial. 
	
	To show that $2)\implies3)$ first note that 
	$R$ contains a minimal left ideal, as $R$ is left artinian. 
	By Lemma~\ref{lem:primoizqmin=>prim}, $R$ is primitive. 

	Now we prove that $3)\implies4)$. If $R$ is primitive, 
	Jacobson's density theorem implies that there exists a division
	ring $D$ such that  
	$R$ is isomorphic to a ring $S$ that is dense in a vector space $V$ over $D$.
	Since $R$ is left artinian, Lemma~\ref{lem:denso_artiniano} implies that  
	$R=\End_D(V)\simeq M_n(D)$, as $\dim_DV<\infty$. 

	Finally, $4)\implies1)$ is trivial, as $M_n(D)$ is simple. 
%	$D$-espacio vectorial $V$ de dimensión finita. Si $u_1,\dots,u_m\in V$ son
%	linealmente independientes sobre $D$, existen $f_1,\dots,f_m\in S$ tales
%	que $f_i(u_i)\ne0$ y $f_i(u_j)=0$ si $i\ne j$. Como los $f_j$ son
%	linealmente independientes sobre $k$, $\dim_DV\leq \dim A$. Luego $A=\End_DV\simeq M_n(D^{\op})$ 
%	por la proposición\dots
\end{proof}

We now prove Artin--Wedderburn theorem. We will assume that our ring
is a unitary left artinian ring. One could prove
Artin--Wedderburn's theorem for arbitrary rings --see for example \cite{MR600654}--  
but when dealing with unitary rings the proof 
is simpler. We will prove
that left artinian semiprimitive unitary rings
are isomorphic to a direct product
of finitely many matrix rings. The idea of the proof goes as follows. 
We know that if 
$R$ is semiprimitive, then $R$ is a subdirect product
of primitive rings, that is  
there exists an injective map
\[
R\to \prod_{i\in I}R/I_i
\]
where each $I_i$ is a primitive ideal. Since $R$ is left artinian, 
the set $I$ will be a finite set. Moreover, 
by Wedderburn's theorem, 
$R/I_i\simeq M_{n_i}(D_i)$ for some division ring $D_i$. Finally,
a sort of non-commutative version of the Chinese remainder theorem
is used to prove that the map is fact surjective. 

\begin{definition}
\index{Ideal!prime}
    An ideal $I$ of $R$ is \textbf{prime} if $xRy\subseteq I$ implies
    $x\in I$ or $y\in I$.
\end{definition}

Note that a ring $R$ is prime if and only if $\{0\}$ is a prime ideal. 
Moreover, 
an ideal $I$ of $R$ is prime if and only if 
the ring $R/I$ is prime. 

\begin{lemma}
    If $R$ is left artinian and $I$ is a primitive ideal, then
    $I$ is prime. 
\end{lemma}

% no necesito hacerlo para artiniano

\begin{proof}
    Since $I$ is primitive, then $R/I$ is primitive. By Wedderburn theorem, 
    $R/I$ is prime and hence $I$ is prime. 
\end{proof}

\begin{theorem}[Artin--Wedderburn]
\label{thm:ArtinWedderburn_version2}
\index{Artin--Wedderburn's theorem}
    Let $R$ be a semiprimitive left artinian unitary ring. Then
    $R\simeq\prod_{i=1}^kM_{n_i}(D_i)$ for finitely many
    division rings $D_1,\dots,D_k$. 
\end{theorem}

We shall need the following lemmas.

\begin{lemma}
\label{lem:primitive=>maximal}
    Let $R$ be a left artinian ring and $I$ be a primitive ideal. 
    Then $I$ is maximal. 
\end{lemma}

\begin{proof}
    If $I$ is a primitive ideal of $R$, then $R/I$ is a primitive ring
    by Lemma \ref{lemma:primitivo}. By Wedderburn's theorem, $R/I$ is
    simple. Thus $I$ is maximal by Proposition \ref{proposition:R/I}. 
\end{proof}

\begin{lemma}
    Let $I_1,\dots,I_k$ be finitely many distinct maximal ideals of $R$. 
    Then $I_2\cdots I_k\not\subseteq I_1$.   
\end{lemma}

\begin{proof}
    Suppose the result is not true and let $k$ be minimal
    such that $I_2\cdots I_k\subseteq I_1$. Since the result is clearly
    true for two distinct maximal ideals, $k\geq3$. Let $I=I_2\cdots I_{k-1}$. 
    Since $I\not\subseteq I_1$, there exists $x\in I\setminus I_1$. Moreover,  
    there exists $y\in I_k\setminus I_1$, as 
    $I_k\ne I_1$. 
    Then 
    $(xR)y\subseteq II_k\subseteq I_1$. Since $I_1$ is prime, 
    it follows that either $x\in I_1$ or $y\in I_1$, a contradiction.   
\end{proof}

\begin{lemma}
    Let $R$ be a left artinian ring. Then $R$ has only 
    finitely many primitive ideals.
\end{lemma}

\begin{proof}
    If $I_1,I_2\dots$ are infinitely many primitive ideals. 
    Since $R$ is left artinian, the sequence 
    $I_1\supseteq I_1I_2\supseteq\cdots$ stabilizes, so there
    exists $n$ such that 
    \[
    I_1I_2\cdots I_n=I_1I_2\cdots I_nI_{n+1}\subseteq I_{n+1},
    \]
    a contradiction to the previous lemma, 
    as each $I_j$ is a maximal ideal. 
\end{proof}

Now we are ready to prove the theorem. 

\begin{proof}[Proof of Theorem \ref{thm:ArtinWedderburn_version2}]
    Let $I_1,\dots,I_k$ be the (distinct) primitive ideals of $R$. 
    We know that each $I_i$ is a maximal ideal. Thus $I_i+I_j=R$ for
    $i\ne j$. Since $R$ is semiprimitive, 
    $I_1\cap\cdots\cap I_k=J(R)=\{0\}$. Let 
    \[
    \varphi\colon R\to \prod_{i=1}^k R/I_i,\quad
    x\mapsto (x+I_1,\dots,x+I_k).
    \]
    Then $\varphi$ is a ring homomorphism with kernel $I_1\cap\cdots\cap I_k=\{0\}$, so
    $\varphi$ is injective. We need to prove that $\varphi$ is surjective. 
    
    We first claim that 
    $I_1+( I_2\cdots I_k) = R$. In fact, 
    since $I_1,\dots,I_k$ are maximal ideals, $I_2\cdots I_k\not\subseteq I_1$. This implies
    that $I_1+(I_2\cdots I_k)$ is an ideal of $R$ that contains $I_1$. Since $I_1$ is maximal, 
    $I_1+(I_2\cdots I_k)=R$. 
    
    Since $I_1+( I_2\cdots I_k) = R$, 
    there exists $x_1\in \prod_{j=2}^kI_j$ such that $1\in x_1+I_1$. Note that
    $x_1=(1+I_1)\cap (I_2\cdots I_k)\subseteq I_j$ for all $j\in\{2,\dots,k\}$. 
    Thus 
    \[
    \varphi(x_1)=(x+I_1,I_2,\dots,I_k)=(1+I_1,I_2,\dots, I_k).
    \]
    Similarly,
    there exists $x_2\in 1+I_2,\dots, x_k\in 1+I_k$ such that 
    \begin{align*}
    \varphi(x_2)&=(I_1,1+I_2,\dots,I_k),\\
    &\vdots\\
    \varphi(x_k)&=(I_1,I_2,\dots,1+I_k).
    \end{align*}
    From this it follows that $\varphi$ is surjective. Each $R/I_i$ 
    is primitive and hence isomorphic to $M_{n_i}(D_i)$ for some 
    $n_i$ and some division ring $D_i$. Therefore
    \[
    R\simeq R/I_1\times\cdots\times R/I_k\simeq \prod_{i=1}^kM_{n_i}(D_i).\qedhere 
    \]
\end{proof}

\topic{Wedderburn's little theorem}

\begin{definition}
	The $n$-th cyclotomic polynomial 
	is defined as the polynomial
	\begin{equation}
		\label{eq:ciclotomico}
		\Phi_n(X)=\prod(X-\zeta),
	\end{equation}
	where the product is taken over all 
	$n$-th primitive roots of one. 
\end{definition}

Some examples:
	\begin{align*}
		&\Phi_2=X-1,\\
		&\Phi_3=X^2+X+1,\\
		&\Phi_4=X^2+1,\\
		&\Phi_5=X^4+X^3+X^2+X+1,\\
		&\Phi_6=X^2-X+1,\\
		&\Phi_7=X^6+X^5+\cdots+X+1.
	\end{align*}

\begin{lemma}
	If $n\in\Z_{>0}$, then
	\[
		X^n-1=\prod_{d\mid n}\Phi_d(X).
	\]
\end{lemma}

\begin{proof}
	Write 
	\[
		X^n-1=\prod_{j=1}^n (X-e^{2\pi ij/n})
		=\prod_{d\mid n}\prod_{\substack{1\leq j\leq n\\\gcd(j,n)=d}}(X-e^{2\pi ij/n})
		=\prod_{d\mid n}\Phi_d(X).\qedhere 
	\]
\end{proof}

\begin{lemma}
	If $n\in\Z_{>0}$, then $\Phi_n(X)\in\Z[X]$.
\end{lemma}

\begin{proof}
	We proceed by induction on $n$. The case $n=1$ is trivial, as 
	$\Phi_1(X)=X-1$. Assume that $\Phi_d(X)\in\Z[X]$ for all $d<n$.
	Then 
	\[
		\prod_{d\mid n,d\ne n}\Phi_d(X)\in\Z[X]
	\]
	is a monic polynomial. Thus $\Phi_n(X)/\prod_{d\mid
	n,d<n}\Phi_d(X)\in\Z[X]$.
\end{proof}

\begin{theorem}[Wedderburn]
\label{thm:Wedderburn} 
	\index{Wedderburn's little theorem}
	Every finite division ring is a field. 
\end{theorem}

\begin{proof}
    Let $D$ be a finite division ring   
	and $K=Z(D)$. Then $K$ is a finite field, say $|K|=q$. 
	We claim that $|q-\zeta|>q-1$ for all $n$-th 
	root of one $\zeta\ne 1$.  In fact, write $\zeta=\cos\theta+i\sin\theta$. Then 
	$\cos\theta<1$ and 
	\[
	|q-\zeta|^2=q^2-(2\cos\theta)q+1>(q-1)^2.
	\]
	
	Note that
	$K$ is a $D$-vector space. Let 
	$n=\dim_KD$.  We claim that $n=1$. If $n>1$, the 
	class equation for the group 
	$D^\times=D\setminus\{0\}$ implies that 
	\begin{equation}
		\label{eq:clases}
		q^n-1=q-1+\sum_{j=1}^m \frac{q^n-1}{q^{d_j}-1},
	\end{equation}
	where $1<\frac{q^n-1}{q^{d_j}-1}\in\Z$ for all $j\in\{1,\dots,m\}$. 
	Since $d^{d_j}-1$ divides $q^n-1$, each $d_j$ divides $n$. In particular,
	~\eqref{eq:ciclotomico} implies that 
	\begin{equation}
		\label{eq:trick_ciclotomico}
		X^n-1=\Phi_n(X)(X^{d_j}-1)h(X)
	\end{equation}
	for some $h(X)\in\Z[X]$. 
	By evaluating~\eqref{eq:trick_ciclotomico} in $X=q$  
	we obtain that $\Phi_n(q)$ divides $q^n-1$ and that $\Phi_n(q)$
	divides $\frac{q^n-1}{q^{d_j}-1}$. By~\eqref{eq:clases}, 
	$\Phi_n(q)$ divides $q-1$. 
	Thus  
	\[
		q-1\geq |\Phi_n(q)|=\prod |q-\zeta|>q-1,
	\]
	as each $|q-\zeta|>q-1$, 
	a contradiction. 
\end{proof}

There are several proofs of Wedderburn's theorem. 
For example, \cite{MR360687} contains a proof that uses only 
elementary linear algebra. In \cite[Chapter 14]{MR896269}
the theorem is proved using group theory. 

\begin{theorem}
\label{thm:division}
    Let $D$ be a division ring of characteristic $p>0$. 
    If $G$ is a subgroup of $D\setminus\{0\}$, then 
    $G$ is cyclic. 
\end{theorem}

We shall need a lemma.

\begin{lemma}
    Let $K$ be a field. 
    Any finite subgroup of $K\setminus\{0\}$ is cyclic. 
\end{lemma}

\begin{proof}
    Let $G$ be a finite subgroup of $K\setminus\{0\}$ and
    $n=|G|$. For a divisor $d$ of $n$, let
    $f(d)$ be the number of elements of $G$ of order $d$. 
    Then 
    \[
    \sum_{d\mid n}f(d)=n.
    \]
    
    We claim that if $d\mid n$ is such that
    $f(d)\ne0$, then $f(d)=\varphi(d)$, where $\varphi$ is
    the Euler function. In fact, if $f(d)\ne 0$, 
    then there exists $g\in G$ such that $|g|=d$. Let 
    $H=\langle g\rangle$ be the subgroup of $G$ generated by $g$.
    Every element of $H$ is a root of the polynomial
    $p(X)=X^d-1\in D[X]$. Since $p(X)$ has at most $d$ roots, 
    $H$ is the set of roots of $p(X)$. In particular, 
    $g^m\in H$ and $|g^m|=d$ if and only if $\gcd(m,d)=1$. Hence
    $f(d)=\varphi(d)$.
    
    The previous claim shows that, in particular, $f(n)=\varphi(n)\ne0$.
    Hence there exists $g\in G$ such that 
    $|g|=n=|G|$ and $G$ is cyclic.
\end{proof}

\begin{proof}[Proof of Theorem \ref{thm:division}]
    Let $F=\sum_{g\in G}(\Z/p)g$. Then $F$ is a finite 
    subring of $D$. Since $D$ is a domain, $F$ is a domain. 
    Let $\alpha\in F\setminus\{0\}$. 
    Then 
    $\{\lambda\alpha:\lambda\in F\}=F$. Since $\lambda\alpha=1$
    for some $\lambda\in F$, $F$ is a division ring. By Wedderburn's 
    theorem, $F$ is a field. Note that $G\subseteq F$. 
    Therefore $G$ is cyclic by the previous lemma. 
\end{proof}

% todo:  (draw a picture of $q$ and $\zeta$ in the complex plane)
% explicar mejor!

% Veamos como corolario una aplicación al último teorema de Fermat en anillos
% finitos. Demostraremos el siguiente resultado:

% \begin{theorem}
% 	Sea $R$ un anillo unitario finito. Entonces para todo $n\geq1$ existen $x,y,z\in
% 	R\setminus\{0\}$ tales que $x^n+y^n=z^n$ si y sólo si $R$ no es un anillo
% 	de división.
% \end{theorem}

% \begin{proof}
% 	Supongamos primero que $R$ es de división. Por el teorema de Wedderburn,
% 	$R$ es entonces un cuerpo finito, digamos $|R|=q$. Como entonces
% 	$x^{q-1}=1$ para todo $x\in R\setminus\{0\}$, se concluye que la ecuación
% 	$x^{q-1}+y^{q-1}=z^{q-1}$ no tiene solución.

% 	Supongamos ahora que $R$ no es de división. Como entonces, en particular,
% 	$R$ no es un cuerpo, $|R|>2$ y luego $x+y=z$ tiene solución en
% 	$R\setminus\{0\}$ (tomar por ejemplo $x=1$, $y=z-1$ y $z\not\in\{0,1\}$).
% 	Como $R$ es finito, $R$ es artiniano a izquierda y entonces el radical de
% 	Jacobson $J(R)$ es nilpotente. Si $J(R)\ne 0$, existe entonces $a\in
% 	R\setminus\{0\}$ tal que $a^2=0$ y luego $a^n=0$ para todo $n\geq2$. En
% 	este caso, la ecuación $x^n+y^n=z^n$ tiene solución en $R\setminus\{0\}$ si
% 	$n\geq 2$ (tomar por ejemplo $x=a$, $y=z=1$). Si $J(R)=0$, entonces, $R$ es
% 	semisimple y luego, por el teorema de Wedderburn,
% 	\[
% 		R\simeq \prod_{i=1}^k M_{n_i}(D_i)
% 	\]
% 	donde los $D_i$ son cuerpos finitos (por ser anillos de división finitos).
% 	Como $R$ no es un cuerpo, hay dos posibilidades: o bien $n_i>1$ para algún
% 	$i\in\{1,\dots,k\}$, o bien $k\geq 2$ y $n_i=1$ para todo
% 	$i\in\{1,\dots,k\}$. En el primer caso, como $M_{n_i}(D_i)$ tiene elementos
% 	no nulos cuyo cuadrado es cero, $R$ también los tiene, y luego, tal como se
% 	hizo antes, vemos que $x^n+y^n=z^n$ tiene solución. En el segundo caso,
% 	$x=(1,0,0,\dots,0)$, $y=(0,1,0,\dots,0)$ y $z=(1,1,0,\dots,0)$ es una
% 	solución de $x^n+y^n=z^n$.
% \end{proof}

\topic{Zsigmondy's theorem}

One of Wedderburn's original proof of Theorem \ref{thm:Wedderburn} 
uses a 
result proved
by Zsigmondy \cite{MR1546236}. Zsigmondy's theorem is 
quite popular in mathematical contests. 

%Let $a>b\geq1$ be such that $\gcd(a,b)=1$. Let $n\geq2$. 
%We say that
%a prime divisor $p$ is a primitive divisor 
%of $a^n-b^n$ if $p\mid a^n-b^n$ and 
%$p\nmid a^k-b^k$ for all $k\in\{1,\dots,n-1\}$. 

\begin{theorem}[Zsigmondy]
\index{Zsigmondy's theorem}
    Let $a>b\geq1$ be such that $\gcd(a,b)=1$ and $n\geq2$. 
    Then there exists a prime divisor of $a^n-b^n$ that does not
    divide $a^k-b^k$ for all $k\in\{1,\dots,n-1\}$ except 
    when $n=2$ and $a+b$ is a power of two or $(a,b,n)=(2,1,6)$. 
\end{theorem}

% https://angyansheng.github.io/blog/an-elementary-proof-of-zsigmondys-theorem

\begin{proof}
    See for example \cite{MR3172590}. 
\end{proof}

We now quickly sketch a proof of Wedderburn
's theorem \ref{thm:Wedderburn} 
based on Zsigmondy's theorem.  

Let $D$ be a division ring of dimension $n$ over $\Z/p$ for a prime
number $p$. Assume 
first that there exists a prime number $q$ such that 
$q\nmid p$ and the order of $p$ modulo $q$ is $n$. Let $x\in D\setminus\{0\}$ be an element of order $q$ and $F$ be the subring
of $D$ generated by $g$. Note that $F$ is a finite-dimensional
$(\Z/p)$-vector space. Let $m=\dim F$. 
Since $g^{p^m-1}=1$, $q$ divides $p^m-1$. Thus $m=n$ and
hence $D=F$ is commutative. 

Assume now that there is no prime number $q$ such that 
$q\nmid p$ and the order of $p$ modulo $q$ is $n$. By Zsigmondy's 
theorem, $n=2$ or $n=6$ and $p=2$. If $n=2$, then 
$D$ is commutative, as it is the subring generated by
any element of $D\setminus \Z/p$. If $n=6$ and $p=2$, then 
the order of 2 modulo 9 is 6. Since $D\setminus\{0\}$ contains
a subgroup of order 9 and all groups of order 9 are abelian, 
we can use the previous argument to complete the proof. 


\topic{Fermat's last theorem in finite rings}

\begin{theorem}
\index{Fermat's last theorem for finite rings}
    Let $K$ be a finite field and $A$ be a finite-dimensional $K$-algebra.
    For $n\geq1$, there exist $x,y,z\in A\setminus\{0\}$ 
    such that $x^n+y^n=z^n$ if and only if 
    $A$ is not a division algebra.
\end{theorem}

\begin{proof}
    Assume first that $A$ is a division algebra. By Wedderburn's theorem, 
    $A$ is a finite field, say $|A|=q$. Then $x^{q-1}=1$ for all $x\in A\setminus\{0\}$.
    Hence $x^n+y^n=z^n$ does not have a solution. 
    
    Conversely, assume that $A$ is not a division algebra. In particular, 
    $A$ is not a field and $|A|>2$. The equation $x+y=z$ has a solution in $A\setminus\{0\}$ (for example, $x=1$, $y=z-1$ and $z\not\in\{0,1\}$ is a solution). Since
    $\dim A<\infty$, the Jacobson radical $J(A)$ is nilpotent. There are two 
    cases to consider. 
    
    If $J(A)\ne\{0\}$, 
    then there exists $a\in A\setminus\{0\}$ such that $a^2=0$. Thus $a^n=0$ 
    for all $n\geq2$. Hence $x^n+y^n=z^n$ has a non-trivial
    solution in $A\setminus\{0\}$ for all $n\geq2$ (for example, take 
    $x=a$ and $y=z=1$).
    
    If $J(A)=\{0\}$, then $A$ is semisimple and 
    $A\simeq\prod_{i=1}^k M_{n_i}(D_i)$ for (finite) division rings $D_1,\dots,D_k$
    and integers $n_1,\dots,n_k$. By Wedderburn's theorem, each $D_i$ is a finite
    field. We consider two possible cases. 
    
    If there exists $i\in\{1,\dots,k\}$ such that $n_i>1$, then
    $M_{n_i}(D_i)$ has non-zero elements such that their squares are zero. Thus 
    there exists $x\in A\setminus\{0\}$ such that $x^2=0$. In particular, 
    $x^n+y^n=z^n$ has a solution. 
    
    If $k\geq 2$, then $x=(1,0,0,\dots,0)$, $y=(0,1,0,\dots,0)$ 
    and $z=(1,1,0,\dots,0)$ is a solution of $x^n+y^n=z^n$.
\end{proof}
