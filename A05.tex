
\section{Dedekind-finite rings}
\label{section:DedekindFinite}

\begin{definition}
\index{Ring!Dedekind-finite}
    Let $R$ be a ring with one. We say that $R$ is \textbf{Dedekind-finite} if 
    for $a,b\in R$ such that $ab=1$ one has $ba=1$. 
\end{definition}

Trivially, commutative rings are Dedekind finite. 

\begin{exercise}
    Give an example of a ring that is not Dedekind-finite. 
\end{exercise}

\begin{exercise}
    Let $R$ be a commutative unitary ring. Prove that $M_n(R)$ is Dedekind-finite. 
\end{exercise}

\begin{proposition}
    Let $R$ be a unitary ring. Then $R$ is Dedekind-finite if and only if the left regular module 
    $\prescript{}{R}{R}$ is Hopfian, that is 
    every surjective $R$-module homomorphism $R\to R$ is injective.
\end{proposition}

\begin{proof}
    Assume first that $R$ is Dedekind-finite. Let $f\colon R\to R$ be a surjective $R$-module homomorphism. There exists $a \in R$ such that $1=f(a)=af(1)$ and hence $f(1)a=1$. Let $x\in\ker f$. Then $0=f(x)a=xf(1)a=x$. Thus $f$ is injective. Conversely, let $a,b\in R$ be such that $ab=1$. Let $f\colon R\to R$, $f(r)=rb$. Then $f$ is surjective, as $f(ra)=(ra)b=r(ab)=r$ for all $r \in R$. By assumption, $f$ is injective. In particular, $ba=1$, as $b=f(1)=f(ba)$. 
\end{proof}

\begin{exercise}
\index{Module!Dedekind-finite}
    Let $R$ be a unitary ring. An $R$-module is said to be \textbf{Dedekind-finite}
    if $M\simeq M\oplus N$ for some $R$-module $N$ implies that $N=\{0\}$. 
    Prove the following statements:
    \begin{enumerate}
        \item $M$ is Dedekind-finite if and only if the ring 
            $\End(M)$ is Dedekind-finite. 
        \item $\prescript{}{R}{R}$ is Dedekind-finite if and only if the ring $R$ is Dedekind-finite. 
    \end{enumerate}
\end{exercise}

\begin{exercise}
    Prove that an $R$-module that is not Dedekind-finite contains a submodule
    of the form $N\oplus N\oplus N\oplus\cdots$ for some non-zero $R$-module $N$. 
\end{exercise}

\begin{exercise}
    Let $R$ be a unitary ring. 
    \begin{enumerate}
        \item Let $a,b\in R$ be such that $ab=1$. For $i,j\geq1$, let $e_{ij}=b^{i-1}a^{j-1}-b^ia^j$. 
            Prove that $e_{ij}e_{kl}=\delta_{jk}e_il$, where $\delta_{ij}$ is the Kronecker function.
            \item Prove that $e_{1n}$ is nilpotent for all $n\geq2$.  
        \item Prove that if $R$ is not Dedekind-finite, then $R$ contains 
            infinitely many nilpotent elements. 
    \end{enumerate}
\end{exercise}

\begin{exercise}
\index{Ring!reversible}
    A ring $R$ is said to be \textbf{reversible} if for all $a,b\in R$ 
    $ab=0$ implies $ba=0$. Prove that a unitary reversible ring is Dedekind-finite. 
\end{exercise}

\index{Ring!reduced}
The previous exercise implies that reduced rings are Dedekind-finite. 

\begin{exercise}
    Prove that every left noetherian ring is Dedekind-finite. 
\end{exercise}

The previous exercise and the Hopkins--Levitzki theorem imply that 
artinian unitary rings are Dedekind-finite. Can you write a 
proof without using the Hopkins--Levitzki theorem? 

\begin{exercise}
    Prove that every algebraic algebra is a Dedekind-finite ring. In particular, 
    finite-dimensional algebras are Dedekind-finite. 
\end{exercise}

\begin{exercise}
    Let $R$ be a unitary ring. 
    Prove that if $R/J(R)$ is Dedekind-finite, then $R$ is Dedekind finite. 
\end{exercise}

Recall that an element $a$ of a ring 
admits a right inverse if there exists an element $b$ such that $ab=1$. 

\begin{exercise}
Let $R$ be a unitary ring and $a\in R$ that admits a right inverse. 
Prove that the following statements are equivalent:
\begin{enumerate}
    \item $a$ admits at least two right inverses. 
    \item $a$ is not a unit. 
    \item $a$ is a left zero divisor. 
\end{enumerate}
\end{exercise}

%\section{Kaplansky's theorem}

Note that an element that admits at least two right inverses cannot be a unit. 

% This is an exercise in Jacobson's basic algebra I, page 91, exercise 7.
% There are other interesting exercises! 

\begin{theorem}[Kaplansky]
\index{Kaplansky's theorem}
    Let $R$ be a unitary ring. If $u\in R$ has more than one right inverse, then 
    $u$ has infinitely many right inverses. 
\end{theorem}

\begin{proof}
    Let $S=\{x\in R:ux=1\}$. Assume that $u$ admits finitely right inverses, that is $S$ is a finite set. Let $s\in S$ and
    $T=\{xy-1+s:x\in S\}$. By assumption, 
    $|S|\geq2$. Note that $T\subseteq S$, as 
    \[
    u(xu-1+s)=(ux)u-u+us=1.
    \]
    Let $f\colon S\to T$, $x\mapsto xu-1+s$. Then $f$ is injective:
    \[
    f(x)=f(y)\implies xu-1+s=yu-1+s\implies xu=yu\implies xy,
    \]
    because $u$ admits a right inverse. Then $|S|\leq |T|\leq |S|<\infty$ and 
    hence $S=T$. In particular, $s\in T$ and therefore 
    $s=xu-1+s$ for some $x\in s$, that is $xu=1$. For $t\in S\setminus\{x\}$, 
    \[
    x=x(ut)=(xu)t=t,
    \]
    a contradiction. 
    Therefore $S$ is an infinite set. 
\end{proof}

The following exercise 
gives another elementary proof of Kaplansky's theorem; see \cite{MR1144353}. 

\begin{bonus}
\label{xca:Rosenholtz}
    Let $R$ be a unitary ring. Prove the following statements:
    \begin{enumerate}
        \item If $a_1,\dots,a_n\in R$ are distinct elements such that $ba_i=1$, then 
            the set 
            \[
            \{a_1-a_i:1\leq i\leq n\}\cup \{1-a_ib:1\leq i\leq n\}
            \]
            contains at least $n+1$ solutions of $bx=0$. 
        \item If $b$ has $n\geq2$ right inverses, then $bx=0$ has at least $n+1$ solutions and hence
            $b$ has $n+1$ right inverses. 
    \end{enumerate}
\end{bonus}

