\chapter{Semisimple algebras}

\begin{definition}
	An \textbf{algebra} (over the field $K$) is a vector space (over $K$) 
	with an associative multiplication $A\times A\to A$ such that
	$a(\lambda b+\mu c)=\lambda(ab)+\mu(ac)$ and
	$(\lambda a+\mu b)c=\lambda(ac)+\mu (bc)$ for all $a,b,c\in A$, and 
	that contains an element $1_A\in A$ such that $1_Aa=a1_A=a$ for all $a\in A$.   
\end{definition}

Note that an algebra over $K$ is a ring $A$ that is a vector space
(over $K$) such that the map $K\to A$, $\lambda\mapsto \lambda1_A$, is injective. 

\begin{definition}
	An algebra $A$ is \textbf{commutative} if $ab=ba$ for all $a,b\in A$. 
\end{definition}

\begin{example}
	The field $\R$ is a real algebra and similarly 
	$\C$ is a complex algebra. Moreover, $\C$ is a real algebra. 
\end{example}

Any field $K$ is an algebra over $K$.

\begin{example}
	Let $K$ be a field. Then $K[X]$, $K[X,Y]$ and $K[[X]]$ are algebras over $K$.
\end{example}

\begin{example}
	If $A$ is an algebra, then $M_n(A)$ is an algebra. 	
\end{example}

The dimension of an algebra is by 
definition the dimension of the 
underlying vector space. 

\begin{definition}
	Let $A$ and $B$ be algebras. A map $f\colon A\to B$ is an \textbf{algebra homomorphism} 
	if $f$ is linear and it is a ring homomorphism.  	
\end{definition}

The map $\C\to\C$, $z\mapsto\overline{z}$, is a ring homomorphism that
is not $\C$-linear, so it is not an $\C$-algebra homomorphism. 

\begin{example}
	Let $G$ be a finite group. The vector space 
	$\C[G]$ with basis $\{g:g\in G\}$
	is an algebra with multiplication
	\[
	\left(\sum_{g\in G}\lambda_gg\right)\left(\sum_{h\in G}\mu_hh\right)
	=\sum_{g,h\in G}\lambda_g\mu_h(gh).
	\] 	
	Note that $\dim\C[G]=|G|$ and
	$\C[G]$ is commutative if and only $G$ is abelian. 
	This is the \textbf{complex group algebra} of $G$. 
\end{example}

Two basic exercises about group algebras.
 
\begin{exercise}
	Let $G$ be a non-trivial finite group. 
	Then $\C[G]$ has zero divisors. 
\end{exercise}

\begin{exercise}
	Let $A$ be an algebra and $G$ be a finite group. 
	If $f\colon G\to\mathcal{U}(R)$ is a group homomorphism, 
	then there exists an algebra homomorphism 
	$\varphi\colon K[G]\to A$ such that $\varphi|_G=f$.   	
\end{exercise}


\begin{definition}
	Let $A$ be an algebra. An (left) \textbf{ideal} of $A$ is an 
	(left) ideal of the ring $A$ that is also a subspace. 
\end{definition}

Let $A$ be an algebra over $K$. If $I$ is a left ideal of the ring $A$, then 
$I$ is a subspace (over $K$), as $\lambda a=\lambda(1_Aa)=(\lambda 1_A)a$ 
for all $\lambda\in K$ and $a\in A$.  
%
%If $I$ is an ideal of $A$, then $A/I$ is an algebra and 
%the canonical map $\pi\colon A\to A/I$, $a\mapsto a+I$, is a surjective
%algebra homomorphism such that $\ker\pi=I$. 

\begin{definition}
	Let $A$ be an algebra. A \textbf{module} over $A$ is a module $M$ of the ring $A$. 
\end{definition}

Note that if $M$ is a module over $A$, 
then $M$ is a vector space 
with $\lambda m=(\lambda 1_A)m$ for all $\lambda\in K$ and $m\in M$. 
 	
\begin{exercise}
	Let $A$ be an algebra and $M$ be a module over $A$. 
	Then $M$ is finitely generated if and only if $M$ is finite-dimensional.
\end{exercise}

An important example of a module is given by the left representation. The 
algebra $A$ is a module over $A$ with the left multiplication. 

\begin{definition}
	Let $A$ be an algebra and $M$ be a module over $A$. Then 
	$M$ is \textbf{simple} if $M\ne\{0\}$ and $\{0\}$ and $M$ 
	are the only submodules of $M$.	
\end{definition}

\begin{definition}
	Let $A$ be a finite-dimensional 
	algebra and $M$ be a finite-dimensional module over $A$. Then 
	$M$ is \textbf{semisimple} if $M$ is a direct sum of 
	finitely many simple submodules.  
\end{definition}

Clearly, a finite direct sum of semisimples is semisimple. 

\begin{lemma}[Schur]
	Let $A$ be an algebra. If $S$ and $T$ are
	simple modules and $f\colon S\to T$ is a non-zero module homomorphism, 
	then $f$ is an isomorphism. 
\end{lemma}

\begin{proof}
	
\end{proof}



 
